% To tar for submission:
%tar cvzf paper.tgz paper.tex paper.bbl <other_figure_files>

\documentclass[twocolumn,prd,amssymb,aps,nofootinbib,showpacs,epsf]{revtex4}
\usepackage[usenames]{color}
\usepackage{ulem}
\usepackage{graphicx}
\usepackage{amssymb}
\usepackage{amsmath}
\usepackage{epstopdf}

\newcommand{\red}{\textcolor{red}}
\newcommand\ben[2]{\textcolor{red}{{#1}\sout{#2}}}
\newcommand\les[2]{\textcolor{blue}{{#1}\sout{#2}}}

%\def\pasp{\ref@jnl{PASP}}               % Publications of the ASP



\begin{document}

\title{Reconstructing the neutron-star equation of state with gravitational-wave detectors from a realistic population of inspiralling binary neutron stars}

\author{Benjamin D. Lackey$^1$, Leslie Wade$^2$}

\affiliation{
$^1$Department of Physics, Princeton University, Princeton, NJ 08544, USA\\
$^2$Department of Physics, University of Wisconsin--Milwaukee, P.O. Box 413, Milwaukee, Wisconsin 53201, USA
}

\begin{abstract}
Gravitational-wave observations of inspiralling binary neutron star systems can be used to measure the neutron-star equation of state (EOS) through the tidally induced shift in the waveform phase that depends on the tidal deformability parameter $\lambda$. Previous work has shown that $\lambda$, a function of the neutron-star EOS and mass, is marginally measurable by Advanced LIGO for a single event when including tidal information up to the merger frequency. In this work, we describe a method for stacking measurements of $\lambda$ from multiple inspiral events to measure the EOS. We use Markov chain Monte Carlo simulations to estimate the parameters of a 4-parameter piecewise polytrope EOS that matches theoretical EOS models to a few percent. We find that, for ``realistic'' event rates ($\sim 40$ binary neutron star inspiral events per year with signal-to-noise ratio $> 8$ in a single Advanced LIGO detector), combining a year of \les{triple-coincident}{} gravitational-wave data with the constraints from causality and recent high mass neutron-star measurements, the EOS above nuclear density can be measured to better than a factor of two in most cases. We also find that in the mass range $1M_\odot$--$2M_\odot$, the neutron-star radius can be measured to $\pm 1$~km and the tidal deformability can be measured to $\pm 1 \times 10^{36}$~g~cm$^2$~s$^2$. The overwhelming majority of this information comes from the loudest $\sim 5$ events. Current uncertainties in the post-Newtonian waveform model, however, lead to systematic errors in the EOS measurement that are as large as the statistical errors, and more accurate waveform models are needed to \les{minimize? reduce? (we'll never remove this error) }{remove} this error.
\end{abstract}

\pacs{
04.40.Dg, %Relativistic stars: structure, stability, and oscillations
26.60.Kp, %Equations of state of neutron-star matter
97.60.Jd  %Neutron stars
}

\maketitle

\ben{new text from Ben}{old text}

\les{new text from Les}{old text}

[[Update the PACS numbers.]]

[[Should you describe briefly where each theoretical EOS came from or is it just not relevant? Maybe people would want to know why the ALF2 EOS is not fit very well but the others are.]]

[[Finish adding discussion of priors into the plots where you added just the constraints from the priors.]]

[[State somewhere what the cutoff frequency is. Did you do this in Section II?]]

[[Replace instances of ``confidence interval'', ``uncertainty'', ``error'', etc. with ``credible interval'' / ``credible region'' where appropriate.]]

[[How do I remove the little space before the underscore in \texttt{lalinference\_mcmc}? \les{It looks like the now use \texttt{LALInferenceMCMC}, according to the new \texttt{LALInference} paper. OCD crisis averted :-) !!}{}]]

[[State the value of $\rho_{\rm nuc}$]]

\section{Introduction}

Observations of neutron stars (NSs), consisting of nuclear matter in the ground state from zero to several times nuclear saturation density, in principle provide an ideal way to measure the nuclear equation of state (EOS) that describes the pressure $p$ as a function of density $\rho$. The most rigorous EOS constraints from NSs have come from measurements of high mass pulsars in compact binary systems, and it is now clear that the EOS must allow for NS masses $\gtrsim 2 M_\odot$~\cite{DemorestPennucciRansom2010, AntoniadisFreireWex2013}. More precise information can be obtained if the NS mass $M$ and radius $R$ are simultaneously measured, and Lindblom demonstrated explicitly that there is a one-to-one map between the relations $R(M)$ and $p(\rho)$~\cite{Lindblom1992}. However, there has been disagreement about the reliability of NS radius measurements. Recent works using multiple NS observations have concluded that the radius of a canonical $1.4M_\odot$ NS, for example, is $9$--$12$~km (for the range of 95\% confidence intervals for three NSs shown in Fig.~1 of Ref.~\cite{OzelBaymGuver2010}), $10.7$--12.5~km (95\% confidence)~\cite{SteinerLattimerBrown2010}, and 7.6--10.4km (90\% confidence)~\cite{GuillotServillatWebb2013}. \les{This sentence confused me.  Is the ``multiple NS observations'' referring to three of the NSs from Fig 1, like you state in the parenthesis?  Is the point that these three papers independently used observations from three of the NSs in Fig 1 (did they use the same three?) to estimate the radius of a $1.4M_\odot$ NS, and all came up with different ranges?}{}

Gravitational-wave (GW) observations of NSs in inspiralling compact binaries, on the other hand, are sensitive only to the density profile of NSs, and are therefore free of the model-dependent uncertainties in the emission and absorption mechanisms that plague electromagnetic observations. In the next few years, science runs will begin for second generation detectors, including the two Advanced LIGO (aLIGO) detectors~\cite{Harry2010} and Advanced Virgo~\cite{Acernese2009}, and design sensitivity will likely be reached by the end of the decade~\cite{AasiAbadieAbbott2013}. In addition, \mbox{KAGRA}~\cite{Somiya2012} (formerly LCGT) and possibly LIGO-India~\cite{IyerSouradeepUnnikrishnan2011} will come online a few years later. The EOS information provided by these detectors comes mainly from tidal interactions during the inspiral of binary neutron star (BNS) and black hole-neutron star (BHNS) systems~\cite{FlanaganHinderer2008}. Because the tidal effect is a strong function of the mass ratio, and observed BNS events are likely to be significantly closer, this tidal interaction is likely to be seen for only BNS systems~\cite{PannaraleRezzollaOhmeRead2011}. Additional information may also come from modes in the post-merger remnant of BNS systems~\cite{StergioulasBausweinZagkouris2011, ClarkBausweinCadonati2014} as well as the damping of quasi-normal modes for low mass, high spin BHNS systems~\cite{LackeyKyutokuShibata2014}. In the \les{quasi-stationary approximation (When I wrote this in our last paper, Jolien said that ``adiabatic approximation'' wasn't exactly right) }{adiabatic approximation}, the quadrupole moment $Q_{ij}$ of one star depends on the tidal field $\mathcal{E}_{ij}$ from the monopole of the other star through the relation $Q_{ij} = -\lambda \mathcal{E}_{ij}$. Here, $\lambda$ is the EOS dependent tidal deformability and is related to the neutron star's dimensionless Love number $k_2$, first calculated in Ref.~\cite{Hinderer2008}, and radius $R$ through the relation $\lambda = \frac{2}{3G} k_2 R^5$, where $G$ is the gravitational constant.
 
The measurability of tidal parameters in BNS systems with aLIGO for gravitational wave frequencies below 450Hz (prior to the last $\sim 20$ GW cycles before merger where higher order corrections to the tidal effect become important) was first examined for polytropic EOSs in~\cite{FlanaganHinderer2008} and for theoretical hadronic and quark matter EOSs in~\cite{HindererLackeyLangRead2010}.  These studies found that for a single aLIGO detector, tidal interactions were only observable during this early inspiral stage for stiff EOSs, NS masses below 1.4~$M_\odot$, and for nearby sources around 100Mpc with typical signal to noise ratios (SNR) $\gtrsim 30$. Damour {\it et al.}, however, found that when also including EOS information from the last $\sim 20$ GW cycles prior to contact, tidal parameters are in fact observable \les{for more likely}{} SNRs of $\sim 16$~\cite{DamourNagarVillain2012}. \les{The highlighted phrase in the previous sentence confused me.  Are you saying that, by including information from the last 20 cycles, the signals are now observable at SNR~16, and that SNR~16 signals are much more likely than SNR~30 signals? }{}

The above studies relied on the Fisher matrix approximation for parameter estimation which assumes the distribution of parameters follows a multivariate Gaussian. Recent works using a fully Bayesian analysis for the three detector LIGO-Virgo network have confirmed that the tidal signal is indeed measurable when including the entire inspiral up to merger~\cite{DelPozzoLiAgathos2013, WadeCreightonOchsner2014}. In particular Del Pozzo {\it et al.}\ used a method to stack tens of observations of a large number of BNS inspiral events to measure $\lambda(M)$~\cite{DelPozzoLiAgathos2013}. By parametrizing $\lambda(M)$ with a linear fit, they found that $\lambda$ could be measured to $\sim 10\%$ at $1.4M_\odot$, but the mass dependence of $\lambda$ could not be found. This was true even though they used an unrealistically large NS mass range of $1M_\odot$--$2M_\odot$ for their simulated population.

In this paper we will demonstrate the advantages of parameterizing the EOS instead of $\lambda(M)$. The main benefit comes from including prior information about the EOS parameters that would be difficult to include for the parameters of a $\lambda(M)$ fit. As will be discussed in Section~\ref{sec:eos}, thermodynamic stability and causality provide the simple constraint that the slope of the EOS must be in the range $0 \le dp/d\epsilon \le c^2$, where $\epsilon$ is the energy density \les{and $c$ is the speed of light (You defined G previously, so maybe define $c$ to be consistent? Your call.)}{}. Additionally, the constraints from the observed $2M_\odot$ NSs can be included by simply rejecting any EOS parameters that lead to a NS maximum mass below $2M_\odot$. We will find that incorporating this information allows us to make significantly stronger statements about the EOS and macroscopic NS properties. And, this is true when using a much narrower, realistic range of NS masses than used by Del Pozzo {\it et al.}~\cite{DelPozzoLiAgathos2013}.

The BNS inspiral signal is almost exactly characterized by the mass, spin, and tidal deformability of each NS, and is therefore free from the intrinsic variability and uncertainties that bias electromagnetic measurements of the mass and radius. Uncertainties in the post-Newtonian waveform model, however, will lead to systematic errors in recovering the tidal deformabilities and, to a lesser extent, masses. Using the Fisher matrix approximation~\cite{Favata2014, YagiYunes2014} as well as a Markov chain Monte Carlo (MCMC) Bayesian analysis~\cite{WadeCreightonOchsner2014}, it was found that the systematic error in $\lambda$ from waveform uncertainties can be as large as the statistical error for a single BNS inspiral observation. When stacking observations the problem becomes worse, as the statistical errors decrease with more observations but the systematic errors do not. We will discuss how systematic errors impact the recovery of the EOS.

We organize the paper as follows. In Section~\ref{sec:waveform} we describe the BNS inspiral waveform and the contribution from tidal interactions, then describe the EOS parameterization and prior information about the EOS in Section~\ref{sec:eos}. We describe our Bayesian method for estimating EOS parameters in Section~\ref{sec:bayes}, and show our results for a range of scenarios in Section~\ref{sec:results}. We then discuss the impact from current uncertainties in the waveform model in Section~\ref{sec:systematic}, and finally summarize our results and compare them to electromagnetic observations in Section~\ref{sec:discussion}.

%\textit{Conventions}: We set $G=c=1$ unless otherwise stated.

\section{The BNS inspiral signal}
\label{sec:waveform}

\red{[[Make sure you cite Damour's papers (two for 4PN energy, and the TaylorXy convention). See his email.]]}

\les{Some citations still needed...  (Ben, you mention an email from Damour.  Do you want to add in those citations?)}{}

\les{Check that notation is consistent with rest of paper.}{}

\les{Should I include $c$'s and $G$'s?}{}

The strain $h$ observed in a GW detector from a wave emitted by a BNS system is related to the two polarizations of the wave and to the detector's antenna beam pattern response:
\begin{equation}
h=F_+ h_+ + F_\times h_\times.
\end{equation}
For a single interferometric detector, the antenna patterns take the form
\begin{eqnarray}
F_+(\theta,\phi,\psi)&=&-\frac{1}{2}(1+\cos^2\theta)\cos(2\phi)\cos(2\psi)\\
\nonumber
&&-\cos(\theta)\sin(2\phi)\sin(2\psi),\\
F_\times(\theta,\phi,\psi)&=&\frac{1}{2}(1+\cos^2\theta)\cos(2\phi)\sin(2\psi)\\
\nonumber
&&-\cos(\theta)\sin(2\phi)\cos(2\psi),
\end{eqnarray}
in a coordinate system where the two arms define the x-axis and the y-axis, and the z-axis is normal to the x-y plane \les{[cite gr-qc/0008066]}{}.  Here, $(\theta, \phi)$ are the sky position in spherical coordinates, and $\psi$ is the polarization angle \les{Should I define what the polarization angle is?}{}.  Assuming the binary components are point particles in a circular orbit and restricting to leading order in amplitude, the two polarizations of the emitted GW seen by an observer at a distance $D$ away are
\begin{eqnarray}
\label{eq:hoftp}
h_+(t)&=&-\frac{4\mu}{D}\left(\frac{1+\cos^2\iota}{2}\right)x(t)\cos\left[2\Phi(t)\right],\\
\label{eq:hoftc}
h_\times(t)&=&-\frac{4\mu}{D}\cos\iota\mbox{ }x(t)\sin\left[2\Phi(t)\right],
\end{eqnarray}
in the time-domain, where $\mu=m_1 m_2/M$ is the reduced mass, $M=m_1+m_2$ is the total mass, $\iota$ is the inclination angle \les{Should I define what the inclination angle is?}{}, $\Phi(t)$ is the orbital phase, and $x=(\pi M f)^{2/3}$ where $f$ is the GW frequency.  Using the Fourier transform convention
\begin{equation}
\tilde{h}(f)=\int_{-\infty}^{+\infty} h(t) e^{-2\pi i f t}dt,
\end{equation}
the stationary-phase approximation can be used to find the two polarizations in the frequency-domain:
\begin{eqnarray}
\label{eq:hoffp}
\tilde{h}_+(f)&=&\sqrt{\frac{5}{24}}\frac{{\cal M}^{5/6}f^{-7/6}}{\pi^{2/3}D}\left(\frac{1+\cos^2\iota}{2}\right)e^{-i\Psi(f)},\\
\label{eq:hoffc}
\tilde{h}_\times(f)&=&-i\sqrt{\frac{5}{24}}\frac{{\cal M}^{5/6}f^{-7/6}}{\pi^{2/3}D}\cos\iota\mbox{ }e^{-i\Psi(f)},
\end{eqnarray}
where the chirp mass ${\cal M}=\eta^{3/5}M$, the symmetric mass ratio $\eta=m_1 m_2/M^2$, $\Psi(f)=2\pi f t(v) - 2\Phi(v) - \pi/4$, and $\tilde{h}_+(f)$ and $\tilde{h}_\times(f)$ are the Fourier transforms of $h_+(t)$ and $h_\times(t)$ respectively.

The orbital phase evolution is described by the following system of ordinary differential equations:
\begin{eqnarray}
\label{eq:dPhidt}
\frac{d\Phi(t)}{dt}&=&\frac{x^{3/2}}{M},\\
\label{eq:dxdt}
\frac{d x(t)}{dt}&=&\frac{-{\cal L}}{E^{\prime}},
\end{eqnarray}
where $E^{\prime}=dE/dx$ is the analytic derivative of the energy $E$, and the luminosity ${\cal L}$ is related to energy via the energy balance requirement $dE/dt=-{\cal L}$.  Given an energy and luminosity, Eqs.~\eqref{eq:dPhidt} and~\eqref{eq:dxdt} can be solved to find $\Phi(t)$ and $x(t)$, which determines the two time-domain GW polarizations in Eqs.~\eqref{eq:hoftp} and~\eqref{eq:hoftc}.

Integrating Eqs.~\eqref{eq:dPhidt} and~\eqref{eq:dxdt} gives the alternate form:
\begin{eqnarray}
\label{eq:int_t}
t(x)&=&t_{\rm ref}+\int_x^{x_{\rm ref}}\frac{E^\prime(y)}{{\cal L}(y)}dy,\\
\label{eq:int_Phi}
\Phi(x)&=&\Phi_{\rm ref}+\int_x^{x_{\rm ref}}\frac{y^{3/2}}{M}\frac{E^\prime(y)}{{\cal L}(y)}dy,
\end{eqnarray}
where $t_{\rm ref}=t(x_{\rm ref})$, $\Phi_{\rm ref}=\Phi(x_{\rm ref})$, and $x_{\rm ref}$ is an arbitrary reference value, following \cite{BuonannoIyerOchsner2009}.  Given an energy and luminosity, Eqs.~\eqref{eq:int_t} and~\eqref{eq:int_Phi} can be solved and used to find $\Psi(f)$, which determines the two frequency-domain GW polarizations in Eqs.~\eqref{eq:hoffp} and~\eqref{eq:hoffc}.

The energy and luminosity of two inspiralling point-particles can be easily solved for in the Newtonian limit of General Relativity.  However, as the system approaches merger, the orbital speed becomes relativistic, and this quasi-Newtonian limit is not enough to accurately describe its waveform.  We therefore use post-Newtonian (PN) theory, which involves expansions of the equations of motion in orders of orbital speed, to push beyond this limit.  Furthermore, compact binaries containing a neutron star noticeably depart from the point-particle approximation at high frequencies and require an additional correction to the energy and luminosity of the system relative to the point-particle terms.  More specifically, tidal interactions contribute to the phase of the orbit by adding a monopole-quadrupole interaction term to the energy.  This additional quadrupole moment adds to the total quadrupole moment of the system, which also increases the luminosity.

As was done in most previous works~\cite{WadeCreightonOchsner2014, DelPozzoLiAgathos2013, Favata2014, YagiYunes2014}, we keep the PN point-particle corrections to the energy and luminosity up to 3.5PN order as well as the leading-order and next-to-leading-order tidal contributions, which enter at the same order as 5PN and 6PN point-particle terms respectively.  The energy and luminosity take the form:
\begin{align}
E &= - \frac{1}{2} M \eta x \left[1 + e_\text{PP-PN}(\eta) + e_\text{Tidal}(\eta, \Lambda_1, \Lambda_2)\right], \\
\mathcal{L} &= \frac{32}{5}\eta^2 x^5 \left[ 1 + l_\text{PP-PN}(\eta) + l_\text{Tidal}(\eta, \Lambda_1, \Lambda_2)\right],
\end{align}
where the point-particle PN terms $e_\text{PP-PN}$ and $l_\text{PP-PN}$ are given in Ref.~\cite{Blanchet2014Review}, and the tidal terms are
\begin{widetext}
\begin{eqnarray}
\label{eq:e_tidal}
e_\text{Tidal} &=& -\frac{9\eta}{2} \left[ (1-3\eta)(\Lambda_1+\Lambda_2) + \sqrt{1-4\eta}(1-\eta)(\Lambda_1-\Lambda_2) \right]x^5 \\
\nonumber
&& - 22\eta \left[ \left(1-4\eta+\frac{19\eta^2}{8}\right)(\Lambda_1+\Lambda_2) + \sqrt{1-4\eta}\left(1-2\eta+\frac{3\eta^2}{8}\right)(\Lambda_1-\Lambda_2) \right]x^6,\\
\label{eq:l_tidal}
l_\text{Tidal}&=&3 \left[ (1-2\eta-4\eta^2)(\Lambda_1+\Lambda_2) + \sqrt{1-4\eta}(1-2\eta^2)(\Lambda_1-\Lambda_2) \right]x^5 \\
\nonumber
&& - \frac{22}{7} \left[ \left(1-\frac{15\eta}{176}+\frac{147\eta^2}{88}-\frac{1547\eta^3}{44}\right)(\Lambda_1+\Lambda_2) \right . \\
\nonumber
&& \left . + \sqrt{1-4\eta}\left(1+\frac{337\eta}{176}+\frac{15\eta^2}{2}-\frac{1085\eta^3}{88}\right)(\Lambda_1-\Lambda_2) \right]x^6,
\end{eqnarray}
\end{widetext}
where $\Lambda_1=\lambda_1/m_1$ and $\Lambda_2=\lambda_2/m_2$.  In order to write Eqs.~\eqref{eq:e_tidal} and~\eqref{eq:l_tidal} in this form, we have also asserted that $m_1 \ge m_2$ in order to express each correction in terms of dimensionless quantities.

As is described in detail in the Appendix of Ref.~\cite{WadeCreightonOchsner2014}, there are several methods to determine the phase evolution of the system from its energy and luminosity, and we use three of them.  In the TaylorT1 method, one numerically integrates Eqs.~\eqref{eq:dPhidt} and~\eqref{eq:dxdt} to find $\Phi(t)$ and $x(t)$. The TaylorT4 method is a slight variation where one re-expands the ratio $-{\cal L}/E^\prime$ on right hand side of Eq.~\eqref{eq:dxdt} then truncates the series at the highest known PN order (here 3.5PN) before numerically integrating to find $\Phi(t)$ and $x(t)$. The TaylorF2 method uses the stationary phase approximation to estimate the Fourier transformed waveform.  Eqs.~\eqref{eq:int_t} and~\eqref{eq:int_Phi} are used to write $\Psi(f)$ in the following form:
\begin{equation}
\label{eq:psi}
\Psi(f)=2\pi f t_{\rm ref} - 2\Phi_{\rm ref} -\frac{\pi}{4}+ 2\int_x^{x_{\rm ref}}\frac{x^{3/2} - y^{3/2}}{M}\frac{E^\prime(y)}{{\cal L}(y)}dy.
\end{equation}
The ratio $E^\prime/{\cal L}$ is then re-expanded and truncated at the highest known PN order (here 3.5PN) before the integral in Eq.~\eqref{eq:psi} is analytically computed to find $\Psi(f)$.  The differences in the PN waveform families are due mostly to the way terms beyond 3.5PN order are truncated.  Thus, the differences in results using one waveform family as the injection and another as templates should give a reasonable estimate of systematic errors due to unknown higher order terms.

The majority of this work uses the TaylorF2 waveform approximant because its fully analytic frequency-domain form makes it the fastest PN waveform to generate.  The complete TaylorF2 strain can be written more compactly in the following form:
\begin{equation}
\tilde{h}(f)=\sqrt{\frac{5}{24}}\frac{{\cal M}^{5/6}}{\pi^{2/3}}\frac{Q(\theta,\phi,\psi,\iota)}{D}f^{-7/6}e^{-i\Psi(f)},
\end{equation}
where
\begin{equation}
Q(\theta,\phi,\psi,\iota) = F_+(\theta,\phi,\psi)\frac{1+\cos^2\iota}{2} - i F_\times(\theta,\phi,\psi)\cos\iota.
\end{equation}
For a single detector, the extrinsic parameters in the amplitude can be grouped together into a single effective distance parameter, which is $D_{\rm eff}=D/|Q(\theta,\phi,\psi,\iota)|$.  The analytic form for the phase $\Psi(f)$ with point-particle terms up to 3.5PN order and leading-order and next-to-leading-order tidal contributions is
\begin{eqnarray}
\Psi(f)&=&2\pi f t_{\rm ref} - 2\Phi_{\rm ref} -\frac{\pi}{4}\\
\nonumber
&&+\frac{3}{128\eta x^{5/2}}\left[1+\psi_\text{PP-PN}(\eta)+\psi_\text{Tidal}(\eta,\Lambda_1,\Lambda_2)\right],
\end{eqnarray}
where the point-particle PN terms $\psi_\text{PP-PN}(\eta)$ are given in \cite{BuonannoIyerOchsner2009}, and
\begin{widetext}
\begin{equation}
\label{eq:psi_tidal}
\psi_\text{Tidal}(\eta,\Lambda_1,\Lambda_2)=\left[-\frac{39}{2}\tilde{\Lambda}(\eta,\Lambda_1,\Lambda_2)\right]x^5+ \left[-\frac{3115}{64}\tilde{\Lambda}(\eta,\Lambda_1,\Lambda_2)+\frac{6595}{364}\sqrt{1-4\eta}\mbox{ }\delta\tilde{\Lambda}(\eta,\Lambda_1,\Lambda_2)\right]x^6,
\end{equation}
where
\begin{eqnarray}
\label{eq:LT}
\tilde{\Lambda}(\eta,\Lambda_1,\Lambda_2)&=&\frac{8}{13}\left[\left(1+7\eta-31\eta^2\right)\left(\Lambda_1+\Lambda_2\right)+\sqrt{1-4\eta}\left(1+9\eta-11\eta^2\right)\left(\Lambda_1-\Lambda_2\right)\right],\\
\label{eq:dLT}
\delta\tilde{\Lambda}(\eta,\Lambda_1,\Lambda_2)&=&\frac{1}{2}\left[\sqrt{1-4\eta}\left(1-\frac{13272}{1319}\eta+\frac{8944}{1319}\eta^2\right)\left(\Lambda_1+\Lambda_2\right) \right .\\
&& + \left . \left(1-\frac{15910}{1319}\eta+\frac{32850}{1319}\eta^2+\frac{3380}{1319}\eta^3\right)\left(\Lambda_1-\Lambda_2\right)\right].
\end{eqnarray}
\end{widetext}
We express the tidal contributions to the TaylorF2 phase in terms of $(\tilde\Lambda,\delta\tilde\Lambda)$ because $\tilde\Lambda$ is comparably more measurable than $\Lambda_1$ and $\Lambda_2$ individually.  For more details regarding this parameterization, see \cite{WadeCreightonOchsner2014,Favata2014}.

We choose to ignore spin and precession in this study for simplicity.  However, it has been shown that ignoring spin in BNS systems can noticeably bias parameter estimation, even for relatively small spin magnitudes \cite{Favata2014}.  Additionally, we choose to end our waveforms when they reach $f_{\rm ISCO}=(6^{3/2}\pi M)$, where $f_{\rm ISCO}$ is the GW frequency of the innermost stable circular orbit (ISCO) of a test particle about a Schwarzschild BH of mass $M$.  For certain EOSs, this termination frequency may be greater than the frequency at which the two components come into contact.  A parallel investigation which studies how spin and the choice of high-frequency cutoffs can affect the estimation of tidal parameters is nearing completion \cite{Nikhef_paper_2} \les{(I don't know how to cite this, but I think they will be publishing this or putting it on the arxiv soon)}{}.

%In this paper we use the post-Newtonian (PN) formalism to construct our waveform model [[as does everyone else]]. In the PN formalism the energy for circular orbits is found from the conservative parts of the Hamiltonian which is now known up to 4PN order [[is this true? Define what xPN order means]] The luminosity is found from a multipolar decomposition of the metric perturbation in the wave zone, which is then matched to the post-Newtonian multipolar decomposition in the near zone \red{[[None of this really matters.]]}. The phase evolution of the orbit is calculated from the assumption of energy balance $dE/dt = -\mathcal{L}$.

%Tidal interactions contribute to the phase of the orbit by adding a monopole-quadrupole interaction term to the energy, and the additional quadrupole moment adds to the total quadrupole moment of the system, increasing the luminosity. As was done in most previous works~\cite{WadeCreightonOchsner2014, DelPozzoLiAgathos2013, Favata2014, YagiYunes2014} we will use the point-particle terms to 3.5PN order as well as the leading and next-to-leading tidal contributions which enter the energy and luminosity at the same order as 5 and 6PN point-particle terms respectively.
%
%The energy and luminosity have the form
%\begin{align}
%E &= - \frac{1}{2} M \eta x \left[1 + e_\text{PP-PN}(\eta) + e_\text{Tidal}(\eta, \Lambda_1, \Lambda_2)\right], \\
%\mathcal{L} &= \frac{32}{5}\eta^2 x^5 \left[ 1 + l_\text{PP-PN}(\eta) + l_\text{Tidal}(\eta, \Lambda_1, \Lambda_2)\right],
%\end{align}
%where the point-particle PN terms $e_\text{PP-PN}$ and $l_\text{PP-PN}$ are given in Ref.~\cite{Blanchet2014Review}, and using the definition $m_1 \ge m_2$, the tidal terms are
%\begin{widetext}
%\begin{align}
%\begin{split}
%e_\text{Tidal} &= -\frac{9\eta}{2} \left[ (1-3\eta)(\Lambda_1+\Lambda_2) + \sqrt{1-4\eta}(1-\eta)(\Lambda_1-\Lambda_2) \right]x^5 \\
%& - 22\eta \left[ \left(1-4\eta+\frac{19\eta^2}{8}\right)(\Lambda_1+\Lambda_2) + \sqrt{1-4\eta}\left(1-2\eta+\frac{3\eta^2}{8}\right)(\Lambda_1-\Lambda_2) \right]x^6,
%\end{split} \\
%\begin{split}
%l_\text{Tidal} &= 3 \left[ (1-2\eta-4\eta^2)(\Lambda_1+\Lambda_2) + \sqrt{1-4\eta}(1-2\eta^2)(\Lambda_1-\Lambda_2) \right]x^5 \\
%& - \frac{22}{7} \left[ \left(1-\frac{15\eta}{176}+\frac{147\eta^2}{88}-\frac{1547\eta^3}{44}\right)(\Lambda_1+\Lambda_2) + \sqrt{1-4\eta}\left(1+\frac{337\eta}{176}+\frac{15\eta^2}{2}-\frac{1085\eta^3}{88}\right)(\Lambda_1-\Lambda_2) \right]x^6.
%\end{split}
%\end{align}
%\end{widetext}
%
%As is described in detail in the Appendix of Ref.~\cite{WadeCreightonOchsner2014}, there are several methods to integrate the energy balance equation $dE/dt = -\mathcal{L}$ to determine the phase evolution from the energy and luminosity, and we will use three of them. In the TaylorT1 method, one numerically integrates the energy balance equations in the form
%\begin{align}
%\label{eq:dpdt}
%\frac{d\phi}{dt} = \frac{x^{3/2}}{M},\\
%\label{eq:dedt}
%\frac{dx}{dt} = \frac{-\mathcal{L}}{dE/dx},
%\end{align}
%where $dE/dx$ is the analytic derivative of $E$. The TaylorT4 method is a slight variation where one re-expands the right hand side of Eq.~\eqref{eq:dedt} then truncates the series at the highest known PN order (here 3.5PN). The TaylorF2 method uses the stationary phase approximation to provide the Fourier transformed waveform. One first analytically integrates Eqs.~\eqref{eq:dpdt}--\eqref{eq:dedt} to obtain the parametric solutions $t(x)$ and $\phi(x)$, then uses the stationary phase approximation to find the Fourier transformed waveform which has the form
%\begin{equation}
%\tilde h(f_{\rm gw}) = A(f_{\rm gw}) e^{i \psi(f_{\rm gw})}.
%\end{equation}
%
%The differences in these waveforms are due mostly to the way terms beyond 3.5PN order are truncated, and thus, the differences in results using these waveforms as templates should give a reasonable estimate of systematic errors due to unknown higher order terms.
%
%The Fourier transform of a restricted TaylorF2 waveform is 
%\begin{align}
%\tilde h_+(f) &= \sqrt{\frac{5}{24}}\frac{1}{\pi^{2/3}} \frac{\mathcal{M}^{5/6} f^{-7/6}}{r} e^{i\psi(f)} \frac{1+\cos^2\iota}{2} \\
%\tilde h_\times(f) &= \sqrt{\frac{5}{24}}\frac{1}{\pi^{2/3}} \frac{\mathcal{M}^{5/6} f^{-7/6}}{r} e^{i[\psi(f) + \pi/2]} \cos\iota,
%\end{align}
%where
%\begin{equation}
%\psi(f) = 2\pi f t_c - \phi_c - \frac{\pi}{4} + \frac{3}{128\eta} \sum_{n=0}^{12} a_n (\pi M f)^{n/3}.
%\end{equation}
%
%The antenna pattern functions are
%\begin{align}
%\begin{split}
%F_+(\theta, \phi; \psi) =& - \frac{1}{2}(1+\cos^2\theta) \cos(2\phi) \cos(2\psi) \\
%& - \cos\theta \sin(2\phi) \sin(2\psi),
%\end{split}\\
%\begin{split}
%F_\times(\theta, \phi; \psi) =& + \frac{1}{2}(1+\cos^2\theta) \cos(2\phi) \sin(2\psi) \\
%& - \cos\theta \sin(2\phi) \cos(2\psi),
%\end{split}
%\end{align}
%where $(\theta, \phi)$ are the sky position in spherical coordinates, and the polarization angle $\psi$ is defined by ...blah... [[See Flanagan Cutler for discussion in terms of orientation of binary angular momentum axis]]
%
%The gravitational wave signal is
%\begin{align}
%\tilde h(f) &=  F_+ h_+(f) + F_\times h_\times(f) \\
%&= \sqrt{\frac{5}{24}}\frac{1}{\pi^{2/3}} \frac{Q(\theta, \phi, \psi, \iota)}{r} \mathcal{M}^{5/6} f^{-7/6} e^{i\psi(f)},
%\end{align}
%where
%\begin{equation}
%Q(\theta, \phi, \psi, \iota) = F_+(\theta, \phi; \psi) \frac{1+\cos^2\iota}{2} + i F_\times(\theta, \phi; \psi) \cos\iota.
%\end{equation}
%For a single detector, this signal can be rewritten in terms of a single effective distance parameter $D_{\rm eff} = r/|Q(\theta, \phi, \psi, \iota)|$, where $D_{\rm eff} \ge r$, so that the signal becomes
%\begin{equation}
%\tilde h(f) =  \sqrt{\frac{5}{24}}\frac{1}{\pi^{2/3}}  \frac{1}{D_{\rm eff}} \mathcal{M}^{5/6} f^{-7/6} e^{i\psi(f)},
%\end{equation}
%and we have absorbed the argument $\arg(Q)$ into the definition of $\phi_c$.

\red{[[Brief statement about how you will be ignoring spin/possible precession, and how that will effect the results. Cite Favata's paper.]]}


\section{The EOS}
\label{sec:eos}

Because the EOS can be found from a full knowledge of the $R(M)$~\cite{Lindblom1992} or $\lambda(M)$ relations~\cite{LindblomIndik2012, LindblomIndik2014}, we can choose to parameterize either $R(M)$, $\lambda(M)$ or $p(\rho)$. As Del Pozzo {\it et al.}\ found by parameterizing $\lambda(M)$ with a linear fit, it is difficult to accurately measure $\lambda(M)$ over a wide range of masses with BNS inspiral observations. Instead they found that $\lambda$ could only be accurately measured for a specific fiducial mass which they chose to be $1.4M_\odot$. This is because, when parameterizing the function $\lambda(M)$, there is little {\it a priori} information about the allowed functional form of $\lambda(M)$. Lattimer and Steiner found similar results for mass and radius observations in Section~4.1 of Ref.~\cite{LattimerSteiner2014} when $R(M)$ was parameterized instead of the EOS. In contrast, it is much simpler to place useful constraints on the functional form of the EOS fit, and this allows one to make significantly stronger statements about the EOS and the behavior of the $\lambda(M)$ and $R(M)$ curves.

\subsection{Current EOS constraints}
\label{sec:eosconstraints}

The {\it a priori} constraints on the EOS that we will use are:
\begin{enumerate}
\item The EOS can be considered known below a density $\rho_0$ defined below, and we use a fixed EOS below that density.
\item The EOS must be a monotonically increasing function $dp/d\epsilon \ge 0$, where $\epsilon$ is the energy density, to satisfy thermodynamic stability.
\item The speed of sound $v_s = \sqrt{dp/d\epsilon}$ must be less than the speed of light, ensuring causality.\footnote{As pointed out by Lindblom~\cite{Lindblom2010}, The EOS of propagating waves is not necessarily the equilibrium EOS where reactions have had time to go to completion.} 
\item The EOS must allow for a maximum NS mass greater than observed masses. Recent pulsar mass measurements in two neutron star-white dwarf binaries have provided convincing evidence that the maximum NS mass is $\gtrsim 2M_\odot$. The pulsar J1614-2230 was found to have a mass of $1.97\pm0.04M_\odot$ ($1\sigma$ confidence)~\cite{DemorestPennucciRansom2010}, and the pulsar J0348+0432 was found to have a mass of $2.01\pm0.04M_\odot$ ($1\sigma$ confidence)~\cite{AntoniadisFreireWex2013}. We will take the $2\sigma$ lower bound of $1.93M_\odot$ on the mass of J0348+0432 as a solid lower bound on the maximum NS mass.
\end{enumerate}

Although other constraints exist, such as the constraint that the maximum mass-shedding (Kepler) frequency be greater than observed spin frequencies, these turn out to be less useful, so we will focus on the ones listed above for simplicity. See Ref.~\cite{ReadLackey2009} for a discussion of other constraints.

\subsection{Choice of parameterization}
\label{sec:eosparam}

We seek a parameterized EOS that will satisfy the above constraints and also have enough freedom to accurately fit the true EOS. Several choices have been presented in the literature. Read {\it et al.}\ examined various types of piecewise polytropes, defined below, and found that a four-parameter fit could adequately match a wide range of theoretical models~\cite{ReadLackey2009}. Steiner {\it et al.}\ used a model with four nuclear parameters around nuclear saturation density and an additional 4 parameters at higher densities to define a 2-piece polytrope with variable dividing densities~\cite{SteinerLattimerBrown2010}, and they also examined several variations~\cite{SteinerLattimerBrown2013}. Finally, Lindblom constructed a spectral expansion of $\gamma(\epsilon) = d\log p / d\log \epsilon$ that appears to converge to tabulated EOS models with fewer parameters than the 4-parameter piecewise polytrope constructed by Read {\it et al.}~\cite{Lindblom2010}. All of these models can be made to satisfy the constraints in Section~\ref{sec:eosconstraints}. However, because the 4-parameter piecewise polytrope~\cite{ReadLackey2009} has been more commonly used in the GW literature, we will focus on it in this paper, and leave a detailed comparison of how the results depend on the choice of parameterization to future work.

For each piece of a piecewise polytrope, the pressure $p$ in the rest-mass density interval $\rho_{i-1} < \rho < \rho_i$ is defined by
\begin{equation}
p(\rho) = K_i \rho^{\Gamma_i},
\end{equation}
where $\Gamma_i$ is the adiabatic index and the constant $K_i$ is chosen such that $p$ is continuous at the boundary $\rho_{i-1}$. For the core of the star, we use 3 polytropes with adiabatic indices $\Gamma_1$, $\Gamma_2$, and $\Gamma_3$, separated by the fixed dividing densities $\rho_1 = 10^{14.7}$~g/cm$^3$ and $\rho_2 = 10^{15}$~g/cm$^3$. These fixed dividing densities were chosen to minimize the least-squares error between the piecewise polytrope fit and a set of 34 tabulated theoretical EOSs~\cite{ReadLackey2009}. In addition to the three adiabatic indices, we also require an additional parameter to determine the overall pressure scaling which we choose to be $p_1 = p(\rho_1)$, the pressure at the first dividing density. We will therefore use as our four EOS parameters $\vec E = (\log(p_1), \Gamma_1, \Gamma_2, \Gamma_3)$ \les{($E$ was used as energy in Section II.  We should change one or the other, I think)}{}.

For the lower density crust of the NS we use a 4-piece polytrope fit to the SLy EOS given by Table~II of Ref.~\cite{ReadLackey2009}\footnote{The values for $K_i$ in Table~II of Ref.~\cite{ReadLackey2009} incorrectly give the pressure in mass-density units $p/c^2$. The values of $K_i$ in the table should therefore be multiplied by $c^2$.}, and we note that the particular choice of the crust EOS effects the results at less than a percent level. We join the fixed crust EOS and parameterized core EOS together at the density $\rho_0$ where the two EOSs intersect. This usually occurs below $\rho_{\rm nuc}$, and we reject any set of EOS parameters where the joining density is below $2.63\times 10^{12}$~g/cm$^3$, where we consider the EOS to be known. Finally, given this EOS, the energy density $\epsilon$, used for solving the stellar structure equations that determine the radius and tidal deformability, can be evaluated by integrating the first law of thermodynamics
\begin{equation}
d\frac{\epsilon}{\rho} = - p d\frac{1}{\rho}.
\end{equation}

We note that for NSs around $1.4M_\odot$ found in BNS systems, the central density is close to or just above the last dividing density $\rho_2 = 10^{15}$~g/cm$^3$ for most EOS parameters~\cite{ReadLackey2009}. Information about $\Gamma_3$ therefore cannot come from BNS inspiral observations alone. However, when used in conjunction with the priors from causality and high mass NS observations, the inspiral will help constrain $\Gamma_3$ and the EOS above $\rho_2$.

In Fig.~\ref{fig:mvs3d}, we show the constraints placed on the parameter space by the requirement of causality and the existence of a NS with a mass of at least $1.93M_\odot$. Although the causality requirement must hold for all densities, we have only assumed that the piecewise polytrope used here is a sufficiently good fit to the true EOS for densities below $\rho_{\rm c, max}$, the central density of the maximum mass NS. Above this density the EOS could take some other form that is not well described by the expression $p = K_3 \rho^{\Gamma_3}$. We have therefore used the weaker constraint of excluding EOS parameters that result in $v_s > c$ below $\rho_{\rm c, max}$ but accepting those parameters if $v_s > c$ above $\rho_{\rm c, max}$ as discussed in Ref.~\cite{ReadLackey2009}. Because most of the mass in a NS at its maximum \les{allowed}{} mass is above $\rho_1$, the maximum mass and maximum speed of sound are mostly independent of $\Gamma_1$, the adiabatic index below $\rho_1$. We therefore show the constraint in the 3 parameter $(\log(p_1), \Gamma_2, \Gamma_3)$ subspace with $\Gamma_1$ fixed at 3.0\red{[[?]]}. 

\begin{figure}[!htb]
\begin{center}
\includegraphics[width=3.2in]{mvs3d.png}
\caption{The joint constraint imposed by causality and the existence of a $1.93M_\odot$ NS. In the dark shaded volume outlined in black, the EOS parameters allow a NS to have $v_s>c$ at some density below $\rho_{\rm c, max}$. In the light shaded volume outlined in red, the EOS parameters don't allow a NS to have a maximum mass of at least $1.93M_\odot$. Also outlined in blue is the surface where the maximum mass is $2.4M_\odot$. A hypothetical $2.4M_\odot$ NS observation would rule out EOS parameters below this surface.}
\label{fig:mvs3d}
\end{center}
\end{figure}

In our analysis below, we will also use the constraints $33.5 \le \log(p_1/(\text{dyne cm}^{-2})) \le 35.5$, $1.4 \le \Gamma_1 \le 5$, $1 < \Gamma_2 \le 5$, and $1 < \Gamma_3 \le 5$. We also use the above requirement that $\rho_0 \ge 2.63\times 10^{12}$~g/cm$^3$ which restricts small values of $\Gamma_1$ when $\log(p_1)$ is large as shown in Fig.~4 of Ref.~\cite{ReadLackey2009}. These boundaries are large enough to incorporate the 34 EOSs considered in Ref.~\cite{ReadLackey2009}, and are also large enough that they have a minimal impact on the results below.

\subsection{Comparison between parameterized and theoretical EOSs}

The piecewise-polytrope fit was originally constructed to match a wide range of theoretical EOSs whether or not they satisfied the then current constraints~\cite{ReadLackey2009}. In this paper we will examine the 7 EOSs that satisfy the causality constraint for densities up to $\rho_{\rm c, max}$ and also have maximum masses above $2M_\odot$. We list these EOSs in Table~\ref{tab:eosfit} along with the piecewise-polytrope parameters that minimize the least-squares residual~\cite{ReadLackey2009}. We also list several NS properties and the associated errors in reproducing them with the fit. In Fig.~\ref{fig:rlofm}, we compare the radius and tidal deformability for these EOSs with their least-squares fits.  The error in $R$ and $\lambda$ above $1M_\odot$ for the EOS fits is usually much less than 10\%, except for the ALF2 EOS.


\begin{table*}[!htb]
\begin{center}
\caption{Comparison between tabulated EOSs and their best fits. The parameters that provide the best fit to the candidate EOSs as well as the residual are given. $p_1$ is in units of dyne/cm$^2$. Values for observables calculated using the tabulated EOSs are also given. $v_{\rm s,max}$ is the maximum speed of sound below $\rho_{\rm c, max}$. $M_{\rm max}$ is the maximum nonrotating mass configuration in units of $M_\odot$. $R_{1.4}$ is the radius of a $1.4~M_\odot$ star in units of km. $\Lambda_{1.4}$ is the tidal deformability of a $1.4~M_\odot$ star. The difference in calculated values for each observable when using the tabulated EOS $(O_{\rm tab})$ versus the best fit parameterized EOS $(O_{\rm fit})$ is calculated with $(O_{\rm fit}/O_{\rm tab}-1)100$. \red{[[Also include central density at 1.4 and maximum mass?]]} }
\begin{tabular}{l|ccccl|rr|rr|rr|rr}
\hline\hline
EOS & $\log(p_1)$ & $\Gamma_1$ & $\Gamma_2$ & $\Gamma_3$ & residual &
$v_{s, \rm max}$ & \% & $M_{\rm max}$ & \% & $R_{1.4}$ & \% & $\Lambda_{1.4}$ & \% \\
\hline
SLy&34.384&3.005&2.988&2.851&0.0020& 0.989&1.41& 2.049&0.02& 11.736&-0.21\\
ENG&34.437&3.514&3.130&3.168&0.015& 1.000&10.71& 2.240&-0.05& 12.059&-0.69\\
MPA1&34.495&3.446&3.572&2.887&0.0081& 0.994&4.91& 2.461&-0.16& 12.473&-0.26\\
MS1&34.858&3.224&3.033&1.325&0.019& 0.888&12.44& 2.767&-0.54& 14.918&0.06\\
MS1b&34.855&3.456&3.011&1.425&0.015& 0.889&11.38& 2.776&-1.03& 14.583&-0.32\\
H4&34.669&2.909&2.246&2.144&0.0028& 0.685&4.52& 2.032&-0.85& 13.774&1.34\\
ALF2&34.616&4.070&2.411&1.890&0.043& 0.642&1.50& 2.086&-5.26& 13.188&-3.66\\
\hline\hline
\end{tabular}
\label{tab:eosfit}
\end{center}
\end{table*}


\begin{figure}[!htb]
\begin{center}
\includegraphics[width=3.2in]{Radiuslambda.pdf}
\caption{Radius and tidal deformability of tabulated EOS models (solid) and the least-squares piecewise-polytrope fits to those tabulated models (dashed). The 20 vertical lines represent the most likely NS masses of the 10 known BNS systems~\cite{Lattimer2012}. Some of these masses, however, have significant uncertainties. The overlapping vertical bands represent the $1\sigma$ uncertainty in the masses of the pulsars J1614-2230 ($1.97\pm0.04M_\odot$)~\cite{DemorestPennucciRansom2010} and J0348+0432 ($2.01\pm0.04M_\odot$)~\cite{AntoniadisFreireWex2013}, both in neutron star-white dwarf binaries.}
\label{fig:rlofm}
\end{center}
\end{figure}


\section{Bayesian inference of EOS parameters}
\label{sec:bayes}

\subsection{Derivation}

\red{[[Make sure all the waveform parameters are explicitly defined in Section II, or else you'll have to define them here.]]}

Given the GW model and EOS fit described in Sections~\ref{sec:waveform} and~\ref{sec:eos}, we will describe here a method for estimating the EOS parameters from a set of GW observations. We use Bayes' theorem to write the posterior density function (PDF) for the universal EOS parameters $\vec E = (\log(p_1), \Gamma_1, \Gamma_2, \Gamma_3)$ that are common to all NSs \les{(The referee made me provide a citation for this type of statement)}{} and the waveform parameters $\vec\theta = (D_L, \alpha, \delta, \psi, \iota, t_c, \phi_c, m_1, m_2, \tilde\Lambda, \delta\tilde\Lambda)$ \les{(Note: Sec II calls $\alpha$ and $\delta$ something else, and uses $D$ instead of $D_L$...)}{} that are unique to each of the $n$ BNS systems as
\begin{equation}
\begin{split}
p(\vec E,\vec\theta_1,\dots,\vec\theta_n | D,\mathcal{H},\mathcal{I})
&= p(\vec E,\vec\theta_1,\dots,\vec\theta_n | \mathcal{H},\mathcal{I}) \\
& \times\frac{ p(D | \vec E,\vec\theta_1,\dots,\vec\theta_n,\mathcal{H},\mathcal{I}) }{ p(D | \mathcal{H},\mathcal{I}) }.
\end{split}
\end{equation}
The data $D = \vec d_1,\dots, \vec d_n$ \les{(Note: I used $D$ as distance in Section II.)}{} is composed of the data streams $\vec d_i(t)$ from the GW detector network for each of the $n$ inspiral events. $\mathcal{H}$ represents the EOS and waveform models, and $\mathcal{I}$ represents \les{the background information for the parameters (I'm not sure what this means)}{}. The quantity $p(\vec E,\vec\theta_1,\dots,\vec\theta_n | \mathcal{H},\mathcal{I})$ is the prior probability density for the EOS and waveform parameters, $p(D | \vec E,\vec\theta_1,\dots,\vec\theta_n,\mathcal{H},\mathcal{I})$ is the likelihood, and the normalization constant $p(D | \mathcal{H},\mathcal{I})$, which we will not need to calculate here, is the evidence. The marginalized PDF for the EOS parameters is then
\begin{equation}
\label{eq:margEOS}
p(\vec E | D,\mathcal{H},\mathcal{I}) = \int d\vec\theta_1 \dots d\vec\theta_n p(\vec E,\vec\theta_1,\dots,\vec\theta_n | D,\mathcal{H},\mathcal{I}).
\end{equation}
This $11n$-dimensional integral is not easily computed, so we will decompose it into blocks that can be individually computed. First, we decompose the waveform parameters into extrinsic parameters $\vec\theta_{\rm ex} = (D_L, \alpha, \delta, \psi, \iota, t_c, \phi_c, \delta\tilde\Lambda)$\footnote{$\delta\tilde\Lambda$ is an intrinsic parameter that in principle provides additional EOS information. However, it is unmeasurable with aLIGO~\cite{WadeCreightonOchsner2014}, so we treat it as a nuisance parameter and group it with the extrinsic parameters when we marginalize over them.} that depend on the observer and intrinsic parameters $\vec\theta_{\rm in} = (m_1, m_2, \tilde\Lambda)$ that are relevant for measuring the EOS such that $\vec\theta = (\vec\theta_{\rm ex}, \vec\theta_{\rm in})$.

The prior can be decomposed into EOS-parameter and waveform-parameter parts using the product rule $p(\vec x, \vec y) = p(\vec x) p(\vec y | \vec x) $ as well as the fact that the $n$ sets of waveform parameters $\vec\theta_i$ are independent
\begin{equation}
p(\vec E,\vec\theta_1,\dots,\vec\theta_n | \mathcal{H},\mathcal{I}) = p(\vec E | \mathcal{H},\mathcal{I}) \prod_{i=1}^n p(\vec\theta_i | \vec E, \mathcal{H},\mathcal{I}).
\end{equation} 
The conditional prior for the waveform parameters of each binary can be further decomposed with the product rule and the fact that the intrinsic and extrinsic parameters are independent
\begin{equation}
\begin{split}
p(\vec\theta_i | \vec E, \mathcal{H},\mathcal{I}) =& p(m_{1i}, m_{2i} | \vec E, \mathcal{H},\mathcal{I}) p(\tilde\Lambda | m_{1i}, m_{2i}, \vec E, \mathcal{H},\mathcal{I})\\
&\times p(\theta_{{\rm ex,}\textcolor{blue}i} | \mathcal{H}, \mathcal{I}).
\end{split}
\end{equation}
\les{I tried to get the equation above, but I don't think I understand it...  Also, was ``ex,i'' in the subscript supposed to be ``ex,$i$''?}{}  Likewise, the likelihood for the $n$ independent BNS observations is
\begin{equation}
p(D | \vec E,\vec\theta_1,\dots,\vec\theta_n, \mathcal{H},\mathcal{I}) = \prod_{i=1}^n p(d_i | \vec\theta_i, \mathcal{H},\mathcal{I}),
\end{equation}
and we have used the fact that the likelihood only depends on the waveform parameters to write $p(\vec d_i | \vec\theta_i, \vec E,\mathcal{H},\mathcal{I}) = p(\vec d_i | \vec\theta_i,\mathcal{H},\mathcal{I})$. (The waveform signal depends on $\vec E$ only through $\tilde\Lambda$ which is already included as a waveform parameter.)

The marginalized PDF (Eq.~\eqref{eq:margEOS}) is now
\begin{equation}
\begin{split}
p(\vec E | D,\mathcal{H},\mathcal{I}) = &\frac{1}{p(D | \mathcal{H},\mathcal{I})} 
\int d\vec\theta_{\rm in, 1} \dots d\vec\theta_{\rm in, n} \left\{ p(\vec E | \mathcal{H},\mathcal{I}) \right. \\
& \times \prod_{i=1}^n \left[p(m_{1i}, m_{2i} | \vec E, \mathcal{H},\mathcal{I}) p(\tilde\Lambda_i | m_{1i}, m_{2i}, \vec E, \mathcal{H},\mathcal{I}) \right.\\
& \left.\left. \times L(\vec d_i; \vec\theta_{\rm in, i}, \mathcal{H}, \mathcal{I}) \right]\right\},
\end{split}
\end{equation}
where we have defined the quasilikelihood for the intrinsic parameters as
\begin{equation}
\label{eq:quasilike}
L(\vec d_i; \vec\theta_{\rm in, i}, \mathcal{H}, \mathcal{I}) = \int d\vec\theta_{\rm ex, i} p(\vec\theta_{\rm ex, i} | \mathcal{H}, \mathcal{I}) p(\vec d_i | \vec\theta_i,\mathcal{H},\mathcal{I}).
\end{equation}
Because $\tilde\Lambda_i$ is a deterministic function of $m_{1i}$, $m_{2i}$ and the EOS parameters,
\begin{equation}
p(\tilde\Lambda_i | m_{1i}, m_{2i}, \vec E, \mathcal{H},\mathcal{I}) = \delta(\tilde\Lambda_i - \tilde\Lambda(m_{1i}, m_{2i}, \vec E)).
\end{equation}
The marginalized PDF finally becomes
\begin{equation}
\label{eq:margEOS2}
\begin{split}
p(\vec E | D,\mathcal{H},\mathcal{I}) =& \frac{1}{p(D | \mathcal{H},\mathcal{I})} \int dm_{11} dm_{21} \dots dm_{1n} dm_{2n} \\
&\times p(\vec E | \mathcal{H},\mathcal{I}) \prod_{i=1}^n \left[ p(m_{1i}, m_{2i} | \vec E, \mathcal{H},\mathcal{I}) \right.\\
& \left. \times L(\vec d_i; \vec\theta_{\rm in, i}, \mathcal{H}, \mathcal{I})|_{\tilde\Lambda_i = \tilde\Lambda(m_{1i}, m_{2i}, \vec E)} \right].
\end{split}
\end{equation}
The problem has now been reduced to computing the quasilikelihood (Eq.~\eqref{eq:quasilike}) for each GW event, then computing Eq.~\eqref{eq:margEOS2}.

\subsection{Likelihood and signal to noise ratio}

The final ingredient we need to evaluate the marginalized PDF is an expression for the likelihood $p(\vec d_i | \vec\theta_i,\mathcal{H},\mathcal{I})$ for each GW event\footnote{In this subsection, which discusses the likelihood for individual GW events, we omit the event index $i$ for clarity.}. In this paper we assume that each detector in the network has stationary, Gaussian noise and that the noise between detectors is uncorrelated. This means that the power spectral density (PSD) $S_n(f)^a$ of the noise $n^a(t)$ in detector $a$ is \red{[[cite]]}
\begin{equation}
\langle \tilde n^a(f) \tilde n^{b*}(f') \rangle = \frac{1}{2} \delta^{ab} \delta (f-f') S_n(f)^a,
\end{equation}
where $\tilde n^a(f)$ is the Fourier transform of the noise of detector $a$, and $\langle\cdot\rangle$ represents an ensemble average. For a GW event with true parameters $\hat\theta$, resulting in the gravitational wave signal $h^a(t; \hat\theta)$, the data stream of detector $a$ will be
\begin{equation}
d^a(t) = n^a(t) + h^a(t; \hat\theta).
\end{equation}

For stationary, Gaussian noise, it is well known that the probability of obtaining the noise time series $n(t)$ is
\begin{equation}
p_n[n(t)] \propto e^{-(n, n)/2},
\end{equation}
where $(a, b)$ is the usual inner product between two time series $a(t)$ and $b(t)$ weighted by the PSD
\begin{equation}
(a, b)=4{\rm Re} \int_0^\infty \frac{\tilde a(f) \tilde b^*(f)}{S_n(f)}\,df.
\end{equation}
If we have a GW model $m(t;\vec\theta)$ that approximates the true signal $h(t;\vec\theta)$, then the likelihood of obtaining the data $d^a(t)$ given the GW model is
\begin{align}
p(d^a | \vec\theta, \mathcal{H}, \mathcal{I}) &= p_n[d^a(t)-m(t; \vec\theta)] \nonumber\\
& \propto e^{-(d^a - m(\vec\theta), d^a - m(\vec\theta))/2}.
\end{align}
If the model $m$ differs from the true signal $h$, a systematic error will be introduced in the recovered waveform parameters. For a network of independent detectors described by the set of time series $\vec d(t)$, the likelihood is then
\begin{equation}
p(\vec d | \vec\theta, \mathcal{H}, \mathcal{I}) = \prod_a p(d^a | \vec\theta, \mathcal{H}, \mathcal{I}).
\end{equation}

The observed signal to noise ratio (SNR) for the data $d^a$ from detector $a$ is Gaussian random variable given by
\begin{equation}
\rho^a = \frac{(d^a, m)}{\sqrt{(m, m)}},
\end{equation}
and has the expectation value (or characteristic SNR) $\langle \rho^a \rangle = \sqrt{(m, m)}$. The SNR for a network of detectors is then $\rho_{\rm net} = \sqrt{\sum_a(\rho^a)^2}$, which has expectation value $\langle \rho_{\rm net} \rangle = \sqrt{\sum_a\langle \rho^a \rangle^2}$.

\subsection{Implementation}

\red{[[You need to explain exactly what your choice of $p(\vec E | \mathcal{H},\mathcal{I})$ and $ p(m_{1i}, m_{2i} | \vec E, \mathcal{H},\mathcal{I})$ are.]]}

We evaluate Eq.~\eqref{eq:margEOS2} for the marginalized PDF in a two-step procedure, similar to that described by Steiner {\it et al.}~\cite{SteinerLattimerBrown2010}, where we first evaluate the quasilikelihood (Eq.~\eqref{eq:quasilike}) for each of the $n$ BNS systems, then evaluate Eq.~\eqref{eq:margEOS2}. In the first step where we evaluate the quasilikelihood $L(\vec d; \vec\theta_{\rm in}, \mathcal{H}, \mathcal{I})$, we use the MCMC sampler \texttt{lalinference\_mcmc} included in the LSC Algorithm Library~\cite{lal} as was implemented in Ref.~\cite{WadeCreightonOchsner2014} and is described in more detail in Ref.~\cite{VanDerSluysRaymondMandel2008}. We use a uniform prior distribution for $\log(D_L)$, $\alpha$, $\sin(\delta)$, $\psi$, $\sin(\iota)$, $t_c$, and $\phi_c$ for the extrinsic parameters. \red{[[The specific range is blah....]]} The priors for the intrinsic parameters of interest $(m_{1}, m_{2}, \tilde\Lambda)$ are not contained in the quasilikelihood (Eq.~\eqref{eq:quasilike}), so we are required to use flat priors. In most cases we used $1M_\odot \le m_2 \le m_1 \le 30M_\odot$ and $0 \le \tilde\Lambda \le 3000$. However, when we injected BNS systems with component masses of $1M_\odot$ in Section~\ref{sec:varymass}, we used $0.5M_\odot \le m_2 \le m_1 \le 30M_\odot$ and $0 \le \tilde\Lambda \le 5000$ so that the posterior was minimally affected by the prior. For $\delta\tilde\Lambda$, we used the range $-500 \le \delta\tilde\Lambda \le 500$\red{[[Is this true?]]}. For each binary event we perform 10 independent runs and test for convergence using the Gelman-Rubin diagnostic~\cite{GelmanRubin1992}, then join the samples. Marginalization over the extrinsic parameters is performed by generating a distribution from the samples of the three intrinsic parameters with a Gaussian kernel density estimator. 

In the second step, we sample the $4+2n$ parameter integrand of Eq.~\eqref{eq:margEOS2} using the affine-invarient ensemble sampler \texttt{emcee}~\cite{ForemanMackeyHogg2013}. We use uniform priors for the EOS parameters with boundaries as described in Section~\ref{sec:eosparam} and additional boundaries from the $1.93M_\odot$ observation and causality constraints as described in Sections~\ref{sec:eosconstraints} and~\ref{sec:eosparam}, and uniform priors for the masses with $1M_\odot \le m_2 \le m_1 \le 3.0M_\odot$ or, when we examine $1M_\odot$ systems, $0.5M_\odot \le m_2 \le m_1 \le 3.0M_\odot$. We again perform 10 runs and test for convergence with the Gelman-Rubin diagnostic before joining the samples. The marginalized distribution $p(\vec E|D, \mathcal{H}, \mathcal{I})$ is then given by a histogram of the samples for $\vec E$.

The evaluation of the quasilikelihood (Eq.~\eqref{eq:quasilike}) for all $n$ events in a population is very computationally expensive, so in some cases we will use the Fisher matrix approximation instead. In the large SNR limit, the difference $\Delta\vec\theta = \vec\theta - \hat\theta$ between the estimated parameters $\vec\theta$ and the true parameters $\hat\theta$ of the binary system obeys a Gaussian distribution~\cite{FinnChernoff1993}. Specifically, for $N$ parameters, the likelihood is
\begin{equation}
\label{eq:gaussianlike}
p(d^a | \vec\theta,\mathcal{H},\mathcal{I}) = \frac{1}{\sqrt{(2\pi)^N \mathrm{det}(\Sigma_{ij})}} e^{-\frac{1}{2} \Sigma^{-1}_{ij} \Delta\theta_i \Delta\theta_j},
\end{equation}
where $\Sigma_{ij}$ is the covariance matrix, and it is given in terms of the Fisher matrix
\begin{equation}
\Gamma_{ij} = (\partial_i h(\hat\theta), \partial_j h(\hat\theta))
\end{equation}
by the relation $\Sigma_{ij}  = \Gamma^{-1}_{ij}$. In the large SNR limit, $\hat\theta$ will be approximately given by the maximum likelihood. When we use the Fisher matrix approximation we will use the mass variables $\ln(\mathcal{M})$ and $\ln(\eta)$ instead of $m_1$ and $m_2$ and flat priors for $\ln(\mathcal{M})$ and $\ln(\eta)$. The quasilikelihood (Eq.~\eqref{eq:quasilike}) marginalized over the extrinsic parameters is simply given by the submatrix of $\Sigma_{ij}$ containing the intrinsic parameters $(\ln(\mathcal{M}), \ln(\eta), \tilde\Lambda)$. However, when we estimate the EOS parameters in Eq.~\eqref{eq:margEOS2}, we always sample the posterior with \texttt{emcee}.

\subsection{Averaged likelihood}

In order to determine the characteristic ability of GW detectors to measure EOS parameters, we will want to average our results for Eq.~\eqref{eq:margEOS2} over many realizations of the detector noise. In Ref.~\cite{NissankeHolzHughes2010} it was shown that averaging over noise realizations is equivalent to setting the noise realization $n^a(t)$ used to generate the data $d^a(t)$ equal to zero, but still using the characteristic noise PSD in the expression for the likelihood. Explicitly, one can define the averaged likelihood as the geometric mean of an ensemble of $M$ identical detectors measuring the same GW event
\begin{equation}
p_{\rm ave}(d^a | \vec\theta, \mathcal{H}, \mathcal{I}) = \left[\prod_{k=1}^M p(d_k^a | \vec\theta, \mathcal{H}, \mathcal{I})\right]^{1/M}.
\end{equation}
Taking $M$ to be large, it was shown that
\begin{equation}
\label{eq:averagedlike}
p_{\rm ave}(d^a | \vec\theta, \mathcal{H}, \mathcal{I}) \propto e^{-\frac{1}{2}(m^a(\theta) - h^a(\hat\theta) | m^a(\theta) - h^a(\hat\theta))}.
\end{equation}
This averaged likelihood will allow us to examine characteristic measurement uncertainties in the EOS parameters independent of the particular noise realization. It will also allow us in Section~VI to separate the effects of systematic errors due to uncertainties in the waveform model from effects due to individual noise realizations.


\section{Results}
\label{sec:results}

In this Section we characterize the ability of the aLIGO--aVirgo network to measure the EOS from a population of BNS inspiral events. We use the zero detuned-high laser power (broadband) configuration~\cite{LIGOnoise} which represents the design sensitivity that may be achieved for the aLIGO detectors by 2019~\cite{AasiAbadieAbbott2013}. The design sensitivity of aVirgo has slightly higher noise and may be achieved by 2021~\cite{AasiAbadieAbbott2013}. For the results here we use the same broadband aLIGO configuration for all three detectors and note that the differences have only a minor quantitative impact on the results.

\subsection{Baseline BNS population}
\label{sec:baseline}

We begin by estimating the EOS parameters for a BNS population with a realistic distribution of masses, number of events, and a moderate EOS, then later examine how our assumptions effect the results. We sample our population as follows:
\begin{enumerate}
\item \textit{Masses.} NSs in most BNS systems are thought to undergo little accretion after their formation, and are therefore found to be in the relatively narrow mass range characteristic of non-accreting NSs. The 10 currently known BNS systems~\cite{Lattimer2012} have most likely NS masses in the range $1.04M_\odot$--$1.53M_\odot$ (some masses have significant uncertainties) and are shown in Fig.~\ref{fig:rlofm}. \"Ozel {\it et al.}~\cite{OzelPsaltisNarayan2012}, for example, modeled the mass distribution of the known BNS systems as a Gaussian and found the most likely values for the mean and standard deviation to be $1.33M_\odot$ and $0.06M_\odot$ respectively. They also found that the mass ratios of the known BNS systems are consistent with each NS being drawn from this distribution independent of its companion. For simplicity, we draw the mass of each NS independent of its companion from a uniform distribution between $1.2M_\odot$ and $1.6M_\odot$. We contrast this with the much larger range of $1M_\odot$--$2M_\odot$ used in the simulations of Del Pozzo {\it et al.}~\cite{DelPozzoLiAgathos2013}. 

\item \textit{Events.} Significant uncertainty exists in the BNS inspiral event rate. Ref.~\cite{LIGORate2010} compiled rate estimates from several population-synthesis models and observations (Table 6 of Ref.~\cite{LIGORate2010}), and summarized the results as follows: a lower 95\% confidence bound of 1 event per Milky Way Equivalent Galaxy per Myr (MWEG$^{-1}$ Myr$^{-1}$), a most likely ``realistic'' value of 100~MWEG$^{-1}$ Myr$^{-1}$, and an upper 95\% confidence bound of 1000~MWEG$^{-1}$ Myr$^{-1}$. This corresponds to GW detection rates of 0.4~yr$^{-1}$, 40~yr$^{-1}$, and 400~yr$^{-1}$ respectively for a single aLIGO interferometer, using the broadband noise curve with a threshold $\text{SNR} > 8$, averaging over sky location and orientation, with NS masses of $1.4M_\odot$, and using a density of 0.0116~MWEG~Mpc$^{-3}$~\cite{LIGORate2010}. 

For the population presented here, we sampled a year of inspiral events using the ``realistic'' event rate of 100~MWEG$^{-1}$ Myr$^{-1}$. \red{[[Is there a missing piece here? Did you draw from a Poisson distribution, or use exactly 100?]]} The masses for each event were sampled as discussed above, and position and orientation were sampled uniformly. For the three-detector network, this resulted in 121 events with \textit{network} $\text{SNR} \ge 8$ \red{[[Why is this so much larger than $40\times\sqrt{3}$?]]}. We performed parameter estimation for the 50 loudest sources with network SNR that ranged from 64.1 down to 10.7. Seven of these sources had network $\text{SNR} > 20$.

\item \textit{EOS.} We used the piecewise-polytrope fit to the MPA1 EOS given in Table~\ref{tab:eosfit} which has a radius and maximum mass roughly in the middle of the range of Table~\ref{tab:eosfit}. Using the fit instead of the tabulated EOS removes the systematic error in the recovered EOS that results from the inexact EOS fit, and we address the effect of the inexact fit in later sections. Given this EOS and the sampled mass, we then calculated the tidal parameter of each NS.

\end{enumerate}

Using zero-noise data, which gives the likelihood averaged over many noise realizations, we show in Fig.~\ref{fig:mpa1fit} the measurability of the EOS, radius, and tidal deformability for the loudest 20 events in our population. The contours represent the 68\% ($1\sigma$), 95\% ($2\sigma$), and 99.7\% ($3\sigma$) confidence. In the left panel, we plot the uncertainty in $\log(p)$ as well as the fractional uncertainty $p/p_{\rm true}$ where we call $p_{\rm true}$ the piecewise-polytrope fit to the MPA1 EOS. We generate these figures as follows: At the density $\rho$, we evaluate $\log(p(\log(p_1), \Gamma_1, \Gamma_2, \Gamma_3, \rho))$ for each set of EOS parameters from the MCMC simulation. We then generate a histogram of $\log(p)$ and use it to evaluate the confidence bounds at that density. We repeat this for each density. The same is done for $p/p_{\rm true}$. In the top right panel, we generate a histogram of radii for each mass $M$ from our MCMC samples. For masses greater than the $1.93M_\odot$ prior, some of the sampled EOS parameters do not allow for a stable NS. In this case, the histogram of radii bifurcates. The unstable samples produce a black hole and the distribution then has a delta function at the Schwarzschild radius $2GM/c^2$. The confidence interval then represents the fraction of MCMC samples that produce a NS in that radius interval \textit{or} a black hole. The bottom right panel shows the confidence intervals for the tidal deformability $\lambda$. For the samples that produce a black hole above $1.93M_\odot$, $\lambda=0$~\cite{BinningtonPoisson2009}.

\begin{figure*}[!htb]
\begin{center}
\includegraphics[width=3.2in]{LALMCMCmpa1FitTaylorF2ZeroNoiseBNS20p.pdf}
\includegraphics[width=3.2in]{LALMCMCmpa1FitTaylorF2ZeroNoiseBNS20Radiuslambda.pdf}
\caption{Uncertainty in the recovered EOS, radius, and tidal deformability for the three-detector aLIGO--aVirgo network. Results are shown for the loudest 20 events with Network SNR from 64.1 down to 13.7. The red, green, and blue regions represent the 68\% ($1\sigma$), 95\% ($2\sigma$), and 99.7\% ($3\sigma$) confidence regions respectively. In the bottom left panel, $p_{\rm true}$ is the pressure of the true ``injected'' EOS, which in this case is the fit to the MPA1 EOS. In the right panels, the vertical line at $1.93M_\odot$ is the minimum mass, set by the prior, where some accepted EOS parameters do not produce a stable NS.}
\label{fig:mpa1fit}
\end{center}
\end{figure*}

Fig.~\ref{fig:mpa1fit} has sharp peaks in the fractional uncertainty $p/p_{\rm true}$ around the variable transition density $\rho_0 \lesssim \rho_{\rm nuc}$ and at the fixed transition densities $\rho_1$ and $\rho_2$ between the polytrope pieces. This is due to the choice of parameterized EOS model. Allowing the transition densities to be additional parameters would likely smooth these features out. Indeed, the EOS fit in Steiner {\it et al.}~\cite{SteinerLattimerBrown2010} which included the transition densities as free parameters did not show such features in the results for $p/p_{\rm true}$. Above the transition density $\rho_2$, the results increasingly underpredict the pressure. This occurs because, although the MPA1 EOS is causal with $v_{\rm s,max}=0.994$ (Table~\ref{tab:eosfit}), the corresponding piecewise-polytrope fit overpredicts $v_{\rm s,max}$ by $\sim 5\%$ and is therefore acausal at high densities. However, the accepted MCMC samples are required to have $v_{\rm s,max} \leq c$, resulting in accepted samples corresponding to smaller pressures. The credible interval is largest at densities below $\sim \rho_{\rm nuc}$ and for corresponding low mass stars where the densities are lower. This results because the bulk of NS matter is above $\sim \rho_{\rm nuc}$, and we included minimal {\it a priori} information on how the core EOS joins onto the lower-density crust EOS. In contrast, Steiner {\it et al.}~\cite{SteinerLattimerBrown2010} parameterized the EOS around $\rho_{\rm nuc}$ in terms of the baryon density and proton fraction with 4 free parameters (Eq.~(33) of Ref.~\cite{SteinerLattimerBrown2010}), and this provides stronger {\it a priori} constraints on the behavior below $\sim \rho_{\rm nuc}$ (Fig.~8 of Ref.~\cite{SteinerLattimerBrown2010}). Overall, it is clear that in some density regions, a significant contribution to the credible interval comes from our choice of EOS parameterization which was not optimized for the purposes here, and not from the sensitivity of the GW detectors. 

We also find that the error in $\lambda$ is smallest in the mass interval $1.2M_\odot$--$1.6M_\odot$ where the BNS masses were drawn from. This is not surprising. However, $\lambda$ can still be measured with comparable accuracy for a much larger range of masses. This is in contrast to the results of Del Pozzo {\it et al.}~\cite{DelPozzoLiAgathos2013} that did not incorporate the additional information about the EOS presented in Section~\ref{sec:eosconstraints}.

Finally, we show how the confidence region depends on the number of events in Fig.~\ref{fig:mpa1fitofn}. The dotted red curve represents the lower limit set by the priors without any BNS inspiral data. This lower limit for the radius of $\sim 10$~km is in mild tension with Guillot {\it et al.}~\cite{GuillotServillatWebb2013} who, combining data from observations of several NSs, found that the NS radius is 7.6--10.4km (90\% confidence). A similar lower bound from the maximum mass and causality constraints was found in Ref.~\cite{HebelerLattimerPethick2013} (dotted curve in the left panel of Fig.~11 of Ref.~\cite{HebelerLattimerPethick2013}). However, by softening the EOS so that $v_s = \sqrt{dp/d\epsilon} = c$ whenever their piecewise polytrope parameterization became acausal, they were able to weaken this constraint, and their lower limit on the radius is $\sim 1$~km less than the one presented here. The upper limit from the prior (not shown here) is a few times larger than the scale of the figures and is set mainly by the causality constraint. This is in contrast to Ref.~\cite{HebelerLattimerPethick2013}. Because they used strong assumptions about the EOS below $\sim \rho_{\rm nuc}$ from chiral effective field theory, they were able to place an upper limit on the NS radius of $\sim 15$~km. Because we use a much less restricted low-density EOS, the pressure at higher densities is allowed to have much larger values, resulting in larger radii.

When data from the BNS inspirals are included, the overwhelming majority of the information about the EOS is obtained from just the loudest 5 events in the population. After the loudest 5 events, including more events does not improve the measurability of the EOS parameters, radius, or tidal deformability.

\begin{figure}[!htb]
\begin{center}
\includegraphics[width=3.2in]{LALMCMCmpa1FitTaylorF2ZeroNoiseperror.pdf}\\
\includegraphics[width=3.2in]{LALMCMCmpa1FitTaylorF2ZeroNoiseRadiuslambda.pdf}
\caption{Same BNS population as Fig.~\ref{fig:mpa1fit}. Contours represent the 95\% credible regions for the loudest 1, 5, and 20 inspiral events. Including quieter events from the population does not improve the results. Also shown is the lower limit on the 95\% credible region from just the maximum mass ($M_{\rm \max} \geq 1.93M_\odot$) and causality priors.}
\label{fig:mpa1fitofn}
\end{center}
\end{figure}


\subsection{Highest known NS mass}

As discussed in Section~\ref{sec:eosconstraints}, the highest mass NSs with rigorous constraints are $\sim 2M_\odot$~\cite{DemorestPennucciRansom2010, AntoniadisFreireWex2013}, and we have used $1.93M_\odot$ as the lower bound on the maximum mass. However, there is also evidence for NSs with higher masses. In particular, a class of pulsars known as black-widows irradiate their companions and generate outflows that are accreted onto the pulsar, significantly increasing the pulsar's mass. Using spectra to determine the radial velocity of the companion, PSR B1957+20 was found to have a mass of $2.40\pm0.12M_\odot$ after correcting for the anisotropic emission of the companion which causes the center of light to lie inward relative to the center of mass. However, considering systematic uncertainties in the model they found a conservative lower limit of $1.66M_\odot$~\cite{VanKerkwijkBretonKulkarni2011}. Another black-widow pulsar, PSR J1311-3430, was found to have a mass of $2.68\pm0.14M_\odot$, but considering similar uncertainties, a lower limit of $2.1M_\odot$ was claimed~\cite{RomaniFilippenkoSilverman2012}. 

In Fig.~\ref{fig:varymmax} we demonstrate that the confirmation of a NS with a lower mass bound of $2.4M_\odot$, for example, would place a significantly tighter lower bound on the pressure for $\rho \gtrsim \rho_1$. The maximum mass of the MPA1 EOS is $2.461M_\odot$ ($2.457M_\odot$ for the piecewise-polytrope fit), so this EOS would almost be ruled out. In contrast, the upper bound on the pressure above $\sim\rho_2$ in Fig.~\ref{fig:varymmax} comes from the causality requirement. (Recall from Fig.~\ref{fig:mvs3d}, the causality constraint restricts large values of $\Gamma_3$ and partially restricts large values of $\Gamma_2$.) Higher mass measurements will therefore not decrease the upper limit on the pressure at the highest NS densities. Similarly, observations of higher mass NSs place tighter lower bounds on the radius and tidal deformability at high masses, but do not improve the upper bound.

\begin{figure}[!htb]
\begin{center}
\includegraphics[width=3.2in]{priormmaxperror.pdf}\\
\includegraphics[width=3.2in]{priormmaxRadiuslambda.pdf}
\caption{Dependence of credible regions on the highest known NS mass. 95\% credible regions are shown for the same 20 BNS systems as in Fig.~\ref{fig:mpa1fit}. The maximum mass prior is set to $M_{\rm max} \geq 1.93M_\odot$ for the red shaded region and $M_{\rm max} \geq 2.4M_\odot$ for the blue shaded region. Also shown are the lower limits on the 95\% credible regions from just the maximum mass and causality priors, with $M_{\rm max} \geq 1.93M_\odot$ (red dashed) and $M_{\rm max} \geq 2.4M_\odot$ (blue dashed).}
\label{fig:varymmax}
\end{center}
\end{figure}


\subsection{Distribution of BNS masses}
\label{sec:varymass}

The density profile in a NS depends on its mass, with more massive stars consisting of denser matter. We thus expect low mass stars to better estimate the lower density EOS, and higher mass stars to better estimate the higher density EOS. We examine how the error in recovering the EOS depends on the range of masses in the BNS population. As before, we use the piecewise polytrope fit to the MPA1 EOS as the true EOS and we examine the results for the loudest 5 BNS systems. The first population is the same as above, using masses uniformly sampled in the range $1.2M_\odot$--$1.6M_\odot$. We also examine three additional sets of masses where the NSs in the population are either all $1.0M_\odot$, all $1.4M_\odot$, or all $1.8M_\odot$. In order to make a direct comparison between these four populations, we hold all of the parameters fixed except for the masses, then adjust the tidal parameter of each system as determined by the fit to the MPA1 EOS. Because we are only examining the loudest 5 systems, adjusting their masses in this range will not push these events above or below the detection threshold. 

In Fig.~\ref{fig:popmass}, we find that when all NSs have masses of either $1.0M_\odot$, $1.4M_\odot$, or $1.8M_\odot$, the uncertainty in pressure is smallest around $10^{14.6}$~g/cm$^3$, $10^{14.7}$~g/cm$^3$, or $10^{14.8}$~g/cm$^3$ respectively. We find similar results for the radius and tidal deformability; the location of the minimum uncertainty scales with the observed masses. Interestingly, the results for masses fixed at $1.4M_\odot$ and for a uniform distribution from $1.2M_\odot$--$1.6M_\odot$ are almost identical, indicating that useful information can be found even if the range of observed masses is very small.

\red{[[Error is smallest at $1.0M_\odot$. Probably because uncertainty in $\tilde\Lambda$ decreases with mass of observed event.]]} 

\begin{figure}[!htb]
\begin{center}
\includegraphics[width=3.2in]{LALMCMCmpa1FitTaylorF2VaryMassperror.pdf}\\
\includegraphics[width=3.2in]{LALMCMCmpa1FitTaylorF2VaryMassRadiuslambda.pdf}
\caption{Dependence of confidence regions on the masses of NSs in the simulated BNS population. Results use the same BNS systems as in Fig.~\ref{fig:mpa1fit}, except the masses are also varied to be all $1.0M_\odot$ (green dot-dashed), all $1.4M_\odot$ (blue dashed), and all $1.8M_\odot$ (magenta). Also shown is the lower limit from just the maximum mass ($M_{\rm \max} \geq 1.93M_\odot$) and causality priors.}
\label{fig:popmass}
\end{center}
\end{figure}


\subsection{Comparison of lalinference\_mcmc with Fisher matrix}
\label{sec:fisher}

In addition to the results above, found by evaluating the quasilikelihood (Eq.~\eqref{eq:quasilike}) with \texttt{lalinference\_mcmc}, we also use the Fisher matrix approximation (Eq.~\eqref{eq:gaussianlike}) to evaluate the quasilikelihood which is significantly faster but only accurate in the limit of large SNR signals. For the Fisher matrix approximation, we used a single detector with the broadband PSD, but scaled the amplitude such that the SNR was equal to the network SNR for the full aLIGO--aVirgo network. The waveform was identical to the one used with \texttt{lalinference\_mcmc}, except we did not use the term in the next-to-leading order tidal correction containing $\delta\tilde\Lambda$, and thus $\tilde\Lambda$ was our only tidal parameter. Because $\delta\tilde\Lambda$ has an insignificant impact on the waveform relative to $\tilde\Lambda$ and is unmeasurable as discussed in Ref.~\cite{WadeCreightonOchsner2014}, this is approximately equivalent to placing an informative prior on $\delta\tilde\Lambda$ (as is done in the \texttt{lalinference\_mcmc} calculation) that is derived from the bounds used for $\Lambda_1$ and $\Lambda_2$. Furthermore, we use flat priors for $\ln(\mathcal{M})$ and $\ln(\eta)$ in the Fisher matrix calculation instead of the flat priors for $m_1$ and $m_2$ used in the \texttt{lalinference\_mcmc} calculation.

In Fig.~\ref{fig:fisher}, we compare the results for calculating the quasilikelihood with \texttt{lalinference\_mcmc} and the Fisher matrix approximation.  We find that even with the differences listed above, the results are still comparable, although the Fisher matrix approximation slightly underestimates the uncertainty in the pressure, radius, and tidal deformability.

\begin{figure}[!htb]
\begin{center}
\includegraphics[width=3.2in]{CompareLALFisherZeroNoiseperror.pdf}\\
\includegraphics[width=3.2in]{CompareLALFisherZeroNoiseRadiuslambda.pdf}
\caption{Same BNS population as Fig.~\ref{fig:mpa1fit}. The SNR in the Fisher matrix calculation is set to match the network SNR of the \texttt{lalinference\_mcmc} calculation. Contours represent 95\% confidence regions.}
\label{fig:fisher}
\end{center}
\end{figure}


\subsection{Dependence on EOS model}

We have examined the measurability of the EOS for the 7 EOSs in Table~\ref{tab:eosfit} which are all causal and have a maximum mass greater than $2M_\odot$. We use the same population of BNS systems as in Section~\ref{sec:baseline} and adjust the tidal deformability according to the EOS. In this case we also use the tabulated EOSs instead of the least-squares fits used above and calculate the confidence regions with the Fisher matrix approximation. The uncertainty in the pressure for these EOSs is shown in Fig.~\ref{fig:perroralleos}, and there are a few features to note. First, if the true EOS has a maximum speed of sound below the central density of the maximum mass NS $v_{\rm s, max}$ that is close to $c$ (SLy, ENG, MPA1, MS1, MS1b), the causality constraint places a useful upper bound on the pressure estimate $p(\rho)/p_{\rm true}(\rho)$ at densities above $\gtrsim \rho_2$. However, for the H4 and ALF2 EOSs where $v_{\rm s, max} \sim 0.6$, the causality requirement only provides a weak constraint on the high-density EOS. Second, we note that softer EOSs (lower pressures) result in stars that are more easily compressed and have higher densities. They will therefore probe higher densities. We find that for the softer EOSs SLy, ENG, and MPA1, the uncertainty in the pressure is minimized at densities $\rho \gtrsim \rho_1$, whereas for the stiffer EOSs MS1, MS1b, H4, and ALF2, the uncertainty in the pressure is minimized at densities $\rho \lesssim \rho_1$.

\begin{figure*}[!htb]
\begin{center}
\includegraphics[width=6.4in]{perroralleos.pdf}
\caption{Dependence of the EOS credible region on the true EOS for the 20 loudest BNS systems. Contours represent the 68\% and 95\% credible regions. The BNS population is the same as in Fig.~\ref{fig:mpa1fit} except for the choice of tabulated EOS. Results were calculated using the Fisher matrix approximation to the quasilikelihood, and the SNR in the Fisher matrix calculation was scaled to match the network SNR as in Fig.~\ref{fig:fisher}. From left to right then top to bottom, the ``true'' EOSs are SLy, ENG, MPA1, MS1, MS1b, H4, and ALF2. Unlike the results above, the ``true'' EOSs used here are the tabulated EOSs, so these results also include the systematic bias from the inexact EOS parameterization as discussed in Section~\ref{sec:systematicEOS}. The corresponding best-fit piecewise polytropes are shown as dashed curves. For each panel, the other EOSs are also shown relative to the ``true'' tabulated EOS. The quantity $p/p_{\rm true}$ is not defined above the highest density in the EOS table, and for SLy, MPA1, and H4, the EOS table ends before the $10^{15.5}$~g/cm$^3$ right limit of each panel.}
\label{fig:perroralleos}
\end{center}
\end{figure*}

The corresponding statistical uncertainty in $R$ and $\lambda$ for these 7 EOSs is shown in Fig.~\ref{fig:structurealleos}. The uncertainty does not depend sensitively on the true values of $R$ and $\lambda$. The full width of the 95\% confidence region is $\sim 1$--2~km for the radius and $\sim 1$--$2\times 10^{36}$~g~cm$^2$~s$^2$ for the tidal deformability over the range $1M_\odot$--$2M_\odot$, and is roughly consistent for all EOSs. For $\lambda$, The minimum uncertainty is always in the mass range $1.2M_\odot$--1.6$M_\odot$ from which the BNS population is sampled.

\red{[[Systematic errors from inexact EOS fit will be discussed below.]]}

\begin{figure*}[!htb]
\begin{center}
\includegraphics[width=6.4in]{constraintsalleos.pdf}
\caption{Dependence of radius and tidal deformability credible regions for the same loudest 20 BNS systems and EOSs used in Fig.~\ref{fig:perroralleos}. Each band represents the 95\% credible region. In the left panels the EOSs are SLy, MPA1, H4, MS1 from bottom to top, and in the right panels they are ENG, ALF2 and MS1b from bottom to top. The lower bound from the mass and causality prior is also shown. \red{[[Should include fit as in the p/ptrue plot for consistency.]]}}
\label{fig:structurealleos}
\end{center}
\end{figure*}


\subsection{Noise realizations}

In the above results, we used the zero-noise averaged likelihood (Eq.~\eqref{eq:averagedlike}) which gives results averaged over individual realizations of the detector noise. To determine how much the noise realizations for the BNS inspiral events in our population effect the results, we inject the inspiral waveforms in our population into five different noise realizations for each of the loudest five events, then use \texttt{lalinference\_mcmc} to evaluate the quasilikelihood. Fig.~\ref{fig:noise} shows the estimates of the EOS, radius, and tidal deformability of each of our five different noise realizations for the population. For the pressure, the lower limits are roughly the same except in the region $10^{14.5}$--$10^{14.9}$~g/cm$^3$. This results from the fact that much of the lower bound is determined by the prior and the choice of EOS parameterization.

\begin{figure}[!htb]
\begin{center}
\includegraphics[width=3.2in]{LALMCMCmpa1FitTaylorF2CompareNoiseperror.pdf}\\
\includegraphics[width=3.2in]{LALMCMCmpa1FitTaylorF2CompareNoiseRadiuslambda.pdf}\caption{Same BNS population and GW detector network as Fig.~\ref{fig:mpa1fit}. Contours represent 95\% credible regions for the loudest 5 events. The different contours correspond to different noise realizations.}
\label{fig:noise}
\end{center}
\end{figure}


\section{Systematic error}
\label{sec:systematic}

The presentation of statistical errors above assumed an exact waveform model and a parameterized EOS that exactly fits the true EOS. Errors in the waveform model from unknown higher order terms or unmodeled effects will introduce a systematic bias in the recovered tidal parameter for each BNS system. In addition, a parameterized EOS that does not sufficiently fit the true EOS will bias the recovered EOS, as well as the $R(M)$ and $\lambda(M)$ curves.

\subsection{Waveform model}

At present, the post-Newtonian inspiral waveform is known to 3.5PN order in the point-particle terms for all powers of $\eta$, but the leading EOS-dependent tidal term enters at the same order as 5PN point-particle terms. Failing to include the unknown 4PN and higher-order point-particle terms can therefore bias the recovered parameters. As demonstrated in Fisher matrix studies~\cite{Favata2014, YagiYunes2014} and in an MCMC study~\cite{WadeCreightonOchsner2014}, the systematic error in the tidal parameter $\tilde\Lambda$ from the current waveform uncertainty is as large as the statistical error for aLIGO. 

To examine the effect that uncertainty in the waveform model has on the recovered EOS, we will use a similar analysis to these previous works. We use variations in the current PN waveforms which, when expanded, agree at 3.5PN order as an estimate of the error in the waveform model. The waveform variations we use, described in Section~\ref{sec:waveform}, are the TaylorF2, TaylorT1, and TaylorT4 waveforms. We injected these three waveform variations into zero-noise data, then used the TaylorF2 waveform as the template to estimate the waveform parameters $\vec\theta$ with \texttt{lalinference\_mcmc}. In Fig.~\ref{fig:systematic} we show the bias in the recovered EOS, radius, and tidal parameter that results from the waveform uncertainty. The ordering is consistent with that of Ref.~\cite{WadeCreightonOchsner2014}; injecting the TaylorT1 waveform and recovering with the TaylorF2 waveform overestimates the tidal parameter while injecting the TaylorT4 waveform and recovering with the TaylorF2 waveform underestimates the tidal parameter. Likewise, injecting the TaylorT1 waveform overestimates the true pressure and radius, while the TaylorT4 waveform underestimates the true pressure and radius. Overall, failing to include the correct 4PN and higher point-particle terms can lead to a bias that is in some cases larger than the 95\% statistical uncertainty.

\begin{figure}[!htb]
\begin{center}
\includegraphics[width=3.2in]{LALMCMCmpa1FitCompareF2T1T4ZeroNoiseperror.pdf}\\
\includegraphics[width=3.2in]{LALMCMCmpa1FitCompareF2T1T4ZeroNoiseRadiuslambda.pdf}
\caption{Systematic errors in the recovered EOS due to uncertainty of the correct gravitational waveform model. TaylorF2, TaylorT1, and TaylorT4 waveforms were injected into the data, and the parameters were recovered with the TaylorF2 waveform. The 95\% credible regions are shown for the loudest 5 systems.}
\label{fig:systematic}
\end{center}
\end{figure}

As in Refs.~\cite{Favata2014, YagiYunes2014, WadeCreightonOchsner2014}, we have focused on the uncertainties in the point-particle description. However, other matter effects in addition to the $\ell=2$ electric tidal interaction used here may also need to be accounted for. Ref.~\cite{Yagi2014}, for example, has calculated the correction to the PN waveform from the electric tidal deformability for all values of $\ell$ as well as from the $\ell=2$ magnetic tidal deformability in addition to the $\ell=2$ electric tidal deformability used here. These effects are small but will lead to a fractional error in the recovered parameters if not properly included. In addition, the amplification of the tidal deformability that occurs due to resonance when the GW frequency approaches each NS's f-mode frequency also leads to a small fractional correction that will need to be accounted for~\cite{FlanaganHinderer2008}. 

As an alternative to the PN approximation, the effective one body (EOB) formalism, which uses various techniques to re-sum the PN series, may converge more rapidly to the true binary waveform. EOB waveforms have been shown to accurately reproduce numerical binary black hole (BBH) waveforms. For example, Damour {\it et al.}~\cite{DamourNagarBernuzzi2013} compared a recent EOB implementation with nonspinning BBH simulations of the last $\sim 30$ GW cycles and found a phase difference of $<0.1$~radians after fitting the unknown 5PN contribution in the EOB radial potential to the numerical data. Since the tidal contribution to the waveform over this same interval is usually more than a radian, the current EOB waveform may be accurate enough for removing systematic errors due to uncertainties in the point-particle model. Tidal interactions have also been calculated in the EOB formalism for the first few multipoles to 2PN order in the EOB radial potential~\cite{BiniDamourFaye2012}, and comparisons with numerical BNS simulations have shown that EOB waveforms, after calibration, are consistent with the numerical waveforms~\cite{BernuzziNagarThierfelder2012, HotokazakaKyutokuShibata2013}. Unfortunately, due to difficulties from hydrodynamics, BNS codes do not yet have the same level of accuracy as BBH codes.

%Ultimately, the analytic expressions for the BNS waveform need to be either confirmed with numerical simulations or calibrated with them. This is a significant challenge since the phase error of BNS codes is not yet at the same level of accuracy as BBH codes. Fortunately, although the tidal contribution accumulates slowly, about half of the phase accumulation occurs during the last $\sim 20$ GW cycles before merger, where numerical simulations can be used to calibrate the waveform.

In our analysis we injected waveforms with zero NS spin and zero eccentricity and used a non-spinning, non-eccentric waveform template to recover the parameters. While NSs in known BNS systems have dimensionless spins of $|\chi| \lesssim 0.02$, not including the spin terms in the template can lead to systematic errors in $\tilde\Lambda$ that are greater than the statistical errors if the NSs have spins of $|\chi| \gtrsim 0.03$~\cite{Favata2014}. Likewise, the systematic error in $\tilde\Lambda$ from not including the eccentricity terms will be greater than the statistical error if the BNS system has an initial eccentricity at 10~Hz of $e_0 \gtrsim 0.003$, and this may occur for a small fraction of BNS systems formed in dense stellar environments~\cite{Favata2014}. These terms will need to be included in future studies.

Finally, in order to measure the NS EOS with BNS inspiral observations, one needs to know that one is observing a BNS inspiral. Attempting to recover the tidal parameter with a BNS waveform template, when one has actually observed a BHNS or BBH inspiral, will significantly bias the results. One can positively identify a BH if its mass or spin is large enough because NSs have a more restricted range of allowed masses and spins ($M \lesssim 3.2M_\odot$ to satisfy causality~\cite{RhoadesRuffini1974} and dimensionless spin $| \chi | \lesssim 0.7$ to satisfy the Kepler constraint~\cite{CookShapiroTeukolsky1994}). Furthermore, NSs in known BNS systems, have masses in the range $1.0M_\odot \lesssim M \lesssim 1.6M_\odot$ shown in Fig.~\ref{fig:rlofm} and spins $|\chi| \lesssim 0.02$~\cite{Favata2014}.  Restricting their analysis to aligned-spin systems, Hannam {\it et al.}~\cite{HannamBrownFairhurst2013} found that it would be difficult to distinguish between NSs and BHs in binaries except for the loudest few percent of signals. However, precession can break the mass-ratio--spin degeneracy, and using precessing waveforms, Chatziioannou {\it et al.}~\cite{ChatziioannouCornishKlein2014} found that one can distinguish between NSs and BHs for the majority of detected signals. Fortunately, as shown in Section~\ref{sec:baseline}, the majority of EOS information comes from the loudest few events, so even if one cannot determine if the weakest signals come from a BBH, BHNS, or BNS event, this will not significantly effect one's ability to measure the EOS.

\subsection{EOS fit}
\label{sec:systematicEOS}

In addition to errors in the waveform model, the choice of EOS parameterization can also effect the measurement of the EOS, and if unable to sufficiently capture complex behavior in the true EOS, the parameterization will introduce systematic errors. As shown in Fig.~\ref{fig:perroralleos}, the piecewise-polytrope fit can usually reproduce the tabulated EOS to a few percent. However, at the lower and upper density regions, the fit becomes worse. In addition, for the ALF2 EOS, there is a significant change in the EOS around $10^{14.5}$~g/cm$^3$ which is not modeled well by the fixed polytrope in that density interval. As a result, although the statistical error is small, the offset from the tabulated ALF2 EOS around $10^{14.5}$~g/cm$^3$ is larger than the 95\% credible interval for the statistical error. For EOSs such as ALF2, the 4-parameter piecewise polytrope used here will not be an appropriate fit. As found in Table~\ref{tab:eosfit}, it has the largest residual of the 7 EOSs by more than a factor of 2, mainly due to the behavior around $10^{14.5}$~g/cm$^3$. 

\red{[[Discuss corresponding effect for reconstructing R and lambda from Fig 9.]]}

The original intent of the 4-parameter piecewise polytrope was to provide a reasonable fit to a wide range of EOS models with a small number of parameters that have an intuitive meaning~\cite{ReadLackey2009}. However, it appears that for directly measuring the EOS with inspiral observations, a more sophisticated EOS will be needed. Two possibilities mentioned in Section~\ref{sec:eosparam}, include the parameterization in Ref.~\cite{SteinerLattimerBrown2010} which uses variable density intervals as well as the spectral fit of Ref.~\cite{Lindblom2010}. However, the optimal parameterization for recovering the EOS and useful nuclear parameters from GW data remains an open question.

\subsection{Other potential sources of error}

In addition to the systematic errors from the waveform model and EOS fit which can be reduced with better modeling, other effects that could interfere with the measurement of $\tilde\Lambda$ and the EOS have been suggested. Using the unipolar inductor model of Goldreich and Lynden-Bell~\cite{GoldreichLyndenBell1969}, it was suggested that interactions between a NS with a large magnetic dipole and its NS companion could create a torque on the binary that could potentially lead to a phase shift in the waveform of a few cycles~\cite{Piro2012}. If true, this would likely contaminate the measurement of tidal interactions. However, as pointed out by Lai~\cite{Lai2012}, the current in the circuit between the binary pair will generate a toroidal magnetic field that breaks the circuit leading to a maximum current. Lai then found that this upper limit is orders of magnitude too small to effect the waveform of the binary.

The tidal interaction discussed so far in this paper assumes the tidal field changes adiabatically. However, if the frequency of the tidal field variation (a multiple of the orbital frequency) approaches the resonant frequency of the various NS oscillation modes, the mode can be driven to large amplitudes. If the mode couples strongly to the tidal potential, this can lead to a phase shift as the binary passes through that resonance during the inspiral. This effect was examined for the g-modes, f-modes, and r-modes of nonspinning and spinning NSs~\cite{Lai1994, HoLai1999}. In general, the g-modes and r-modes will lead to a phase shift less than 0.1~radians, with maximum values for the largest spins and radii. The f-mode that has a small impact on the last few cycles, as discussed above, can lead to an appreciable effect at lower GW frequencies only if the NS is spinning at several hundred Hz. 

%[[Mode coupling. Tidally deformed NSs are unstable to the coupling of high order p-modes and g-modes~\cite{WeinbergArrasBurkart2013}. Accounting for three- and four-mode couplings, this instability cancels~\cite{VenumadhavZimmermanHirata2014}.]]

Finally, detector calibration errors will also impact the recovery of parameters. The uncertainty in the amplitude and phase of the Fourier transformed detector output $\tilde d(f)$ due to calibration errors are frequency dependent, and, for initial LIGO, were $\sim 10\%$ for the amplitude and $\sim 0.05$~radians for the phase of $\tilde d(f)$ over the bandwidth of the detector~\cite{VitaleDelPozzoLi2012}. The intrinsic parameters we are interested in here (masses and tidal parameters) are mostly determined by the phase evolution, and under the assumption that calibration errors will be similar for aLIGO, Vitale {\it et al.} found, using a Bayesian analysis, that the uncertainties in the chirp mass and symmetric mass ratio due to calibration errors would be $\sim 20\%$ of a standard deviation for the statistical errors from detector noise~\cite{VitaleDelPozzoLi2012}. For the tidal parameter $\tilde\Lambda$ which contributes a few radians to the GW phase, this $\sim 0.05$~radian calibration error is not likely to dominate over the statistical error, but it should be included in future studies.

\section{Discussion}
\label{sec:discussion}

We have shown that, with a realistic population of BNS inspiral events, the advanced LIGO--Virgo GW network now undergoing construction can provide valuable information about the EOS and NS properties such as the radius and tidal deformability. Typical statistical errors in the pressure will be on the order of 10\% to a factor of 2 from 1--4 times nuclear density. This corresponds to a 95\% credible region of width $\sim 1$--2~km for the radius and $\sim 1$--$2\times 10^{36}$~g~cm$^2$~s$^2$ for the tidal deformability over the mass range $1M_\odot$--$2M_\odot$. These results are in agreement with those of Ref.~\cite{DelPozzoLiAgathos2013} who found that with tens of BNS observations, $\lambda$ could be measured at a reference mass of $1.4M_\odot$ to about 10\%. However, our results show that incorporating additional known EOS information allows us to measure NS properties at other masses as well, and this is true even with a more narrow, realistic range of NS masses.

However, in Section~\ref{sec:systematic} we found that currently unknown high order PN point-particle terms will create a large systematic bias in the recovered EOS that will be larger than the statistical uncertainties. These terms will need to be calculated or fit with numerical simulations to accurately measure the EOS, and significant work remains before the resulting waveform templates can be trusted to be sufficiently accurate. The choice of EOS parameterization can also effect the recovered EOS, and EOSs with complex behavior such as the ALF2 EOS will need to be modeled with more sophisticated EOS parameterizations. 
%Other currently unmodeled behavior such as magnetic interactions or tidally induced resonances with low frequency g-modes are unlikely to have an important effect on the waveform phase.

%[[Significant work remains on the part of numerical and analytical modeling of BNS waveforms to produce accurate templates and in efficiently implementing the resulting waveforms templates for Bayesian parameter estimation, as well as incorporating the known effects (very end of the inspiral until amplitude reaches zero, precessing spin, eccentricity, cosmological redshift, analysis of calibration errors). However, these are not show stoppers.]]

Our results indicate that advanced GW detectors can measure the NS radius with similar statistical errors to those from electromagnetic observations of NSs whose errors are typically in the range of a few km. However, although both electromagnetic and GW observations have systematic errors, the gravitational waveforms from inspiralling BNS systems are an intrinsically cleaner source of data than the electromagnetic emission of NSs. Ultimately, electromagnetic and GW measurements will have to agree, and comparing the results will provide an important consistency check on our understanding of general relativity, electromagnetic radiation from NSs, and the EOS.
 
\acknowledgements

BL sincerely thanks Walter Del Pozzo and Tyson Littenberg for advice, many helpful discussions, and correcting misunderstandings in the implementation of Bayesian techniques, as well as Will Farr, Tjonnie Li, Chris Van Den Broeck, and Andrew Steiner. BL also had several informative discussions at the University of Washington's Institute for Nuclear Theory Program 14-2a. The authors would also like to thank Evan Ochsner and Jolien Creighton for advice and for generating the parameters of the BNS population used here. This work was supported by NSF Grants PHY-1305682, and PHY-\red{xxxxxxx [[Les, fill this in]]}. Simulations were run on the Nemo cluster at the University of Wisconsin--Milwaukee and the Orbital cluster at Princeton University.

\bibliography{paper}
\end{document}  